% Template:     Informe LaTeX
% Documento:    Archivo principal
% Versión:      8.2.5 (01/06/2023)
% Codificación: UTF-8
%
% Autor: Pablo Pizarro R.
%        pablo@ppizarror.com
%
% Manual template: [https://latex.ppizarror.com/informe]
% Licencia MIT:    [https://opensource.org/licenses/MIT]

% CREACIÓN DEL DOCUMENTO

\documentclass[spanish]{article}

% 	spanish, english % Idioma: spanish, english, etc.
% 	letterpaper, oneside
% ]{article}
% INFORMACIÓN DEL DOCUMENTO
\def\documenttitle {Proyecto Final de Carrera}
\def\documentsubtitle {}
\def\documentsubject {Estacionando: una solución al problema de conseguir estacionamiento}

\def\documentauthor {Emanuel Cima - Diego Sarina}
\def\coursename {Proyecto II}
\def\coursecode {}

\def\universityname {Universidad Nacional de Rosario}
\def\universityfaculty {Facultad de Ciencias Exactas, Ingeniería y Agrimensura}
\def\universitydepartment {}
\def\universitydepartmentimage {departamentos/FCEIA}
\def\universitydepartmentimagecfg {height=2cm}
\def\universitylocation {Rosario, Argentina}

\def\imagenuniversidad {departamentos/URN}
\def\imagenuniversidadcfg {height=2cm}


% INTEGRANTES, PROFESORES Y FECHAS
\def\authortable {
	\begin{tabular}{ll}
		Integrantes:
		& \begin{tabular}[t]{l}
    		Emanuel Cima \\
    		Diego Sarina
		\end{tabular} \\
        \\
		Profesor:
		& \begin{tabular}[t]{l}
			Ing. Jose Luis Simon
		\end{tabular} \\
	\\
		\multicolumn{2}{l}{Fecha de entrega: \today} \\
		\multicolumn{2}{l}{\universitylocation}
	\end{tabular}
}



% IMPORTACIÓN DEL TEMPLATE
\input{template}

\usepackage{empheq}

% INICIO DE PÁGINAS
\begin{document}

% PORTADA
\templatePortrait

% CONFIGURACIÓN DE PÁGINA Y ENCABEZADOS
\templatePagecfg

% RESUMEN O ABSTRACT
\section*{Resumen}
El objetivo principal de este proyecto es desarrollar un sistema de estacionamiento inteligente, que combina la tecnología de sensores y la potencia de cálculo en la nube y en el borde (edge computing). Se ha realizado un exhaustivo estudio del estado del arte en esta área, investigando los diferentes tipos de sensores disponibles y su aplicabilidad en entornos de estacionamiento.


Se llevaron a cabo pruebas de campo para evaluar la precisión y confiabilidad de los sensores en la detección de la disponibilidad de espacios de estacionamiento. Estas pruebas permitieron recopilar datos sobre la ocupación de los espacios en tiempo real y evaluar el rendimiento de los sensores en condiciones reales.


Además, se diseñó y desarrolló una aplicación móvil utilizando Flutter, un framework de desarrollo multiplataforma, que permite a los usuarios acceder a la información en tiempo real sobre la disponibilidad de espacios de estacionamiento. La aplicación ofrece funcionalidades como la búsqueda de estacionamiento cercano y la reserva de una plaza en el mismo.


Para garantizar un rendimiento óptimo y una respuesta rápida, se aprovechó la potencia de cálculo tanto en la nube como la computación de borde para el procesamiento de los datos de los sensores. Utilizando una Raspberry Pi se pudieron procesar de una manera eficiente los datos enviados por los sensores a través de LoRa y proporcionar actualizaciones en tiempo real a los usuarios de la aplicación.


En resumen, este proyecto abarcó desde el análisis del estado del arte hasta la construcción de un sistema de estacionamiento inteligente (prototipo). Se investigaron los sensores adecuados, se realizaron pruebas de campo, se desarrolló una aplicación móvil y se implementaron soluciones de computación en la nube y en el edge para ofrecer una experiencia completa al usuario. Finalmente se desarrolló un apartado de posibles mejoras al mismo.

\newpage
\section*{Abstract}
The main objective of this project is to develop an intelligent parking system that combines sensor technology and cloud and edge computing power. A comprehensive study of the state of the art in this area has been carried out, investigating the different types of sensors available and their applicability in parking environments.

Field tests were carried out to evaluate the accuracy and reliability of the sensors in detecting parking space availability. These tests allowed data to be collected on space occupancy in real-time and to evaluate sensor performance under real conditions.

In addition, a mobile application was designed and developed using Flutter, a cross-platform development framework, which allows users to access real-time information on parking space availability. The application offers features such as searching for nearby parking and reserving a space in it.

To ensure optimal performance and fast response, computing power was leveraged both in the cloud and edge computing for processing sensor data. Using a Raspberry Pi, data sent by sensors via LoRa could be efficiently processed and real-time updates provided to application users.

In summary, this project covered everything from state-of-the-art analysis to the construction of an intelligent parking system (prototype). Suitable sensors were investigated, field tests were conducted, a mobile application was developed, and cloud and edge computing solutions were implemented to offer a complete user experience. Finally, a section of possible improvements was developed.


% TABLA DE CONTENIDOS - ÍNDICE
\templateIndex

% CONFIGURACIONES FINALES
\templateFinalcfg

% ======================= INICIO DEL DOCUMENTO =======================

\newpage
\section{Introducción}
El creciente aumento en el número de vehículos en nuestras ciudades ha planteado un desafío significativo para los conductores: encontrar un lugar adecuado para estacionar. El estacionamiento se ha convertido en una tarea frustrante y estresante que consume tiempo y afecta negativamente la eficiencia y comodidad del transporte urbano. Este problema no solo afecta a los conductores individuales, sino que también tiene un impacto en la congestión del tráfico, la contaminación ambiental y la calidad de vida en general.

En este contexto, surge la necesidad de buscar soluciones innovadoras que aborden eficazmente el problema de conseguir estacionamiento. El presente proyecto de grado, titulado \quotes{Estacionando}, se propone como una respuesta a esta problemática. El objetivo principal de este proyecto es desarrollar una solución integral que facilite a los conductores encontrar y asegurar lugares de estacionamiento de manera eficiente, reduciendo así la congestión y mejorando la experiencia de conducción.

 
% SUB-SECCIÓN
\subsection{Propósito del proyecto}
La movilidad urbana se ha convertido en un tema crucial para la Comisión Europea y muchas naciones del mundo, en su búsqueda por desarrollar modelos de ciudades inteligentes que mejoren la calidad de vida de sus habitantes. En esta visión de las Smart Cities, la Movilidad Inteligente y Sostenible desempeña un papel central.

En un entorno urbano cada vez más poblado y congestionado, se presentan nuevos desafíos para mejorar la movilidad. Según un informe de McKinsey Sustainability \cite{ref1}, se están explorando diversas alternativas para abordar este problema, como la implementación de coches eléctricos, navegación autónoma, políticas de transporte público y servicios de compartición de vehículos, como el carpooling y Uber, entre otros.

Además, según un informe del Observatorio de Movilidad Urbana (Eltis) \cite{ref2}, financiado por la Comisión Europea, es fundamental diseñar e integrar Sistemas de Estacionamientos Inteligentes (Smart Parking Systems) en las políticas de movilidad urbana sostenible. Una gestión eficiente del estacionamiento puede liberar valioso espacio público, hacer que las ciudades sean más atractivas, reducir el tráfico vehicular, aliviar la congestión y disminuir las emisiones de carbono. De acuerdo con un estudio realizado por la División de Ingeniería de Sistemas y el Centro de Ingeniería de Información y Sistemas de la Universidad de Boston \cite{ref3}, se estima que el 30\% de los conductores de las principales ciudades de Estados Unidos invierten en promedio 7.8 minutos diarios en buscar lugares de estacionamiento libres. Además, según el ensayo ``A Reservation-Based Smart Parking System''\cite{ref4}, durante las horas pico, el tráfico generado por los conductores buscando estacionamiento representa el 40\% del tráfico total en las grandes ciudades estadounidenses. En zonas densamente pobladas, la congestión y los retrasos en el tráfico se deben, en parte, a la problemática del estacionamiento.

Es evidente que una Smart City debe abordar el problema del estacionamiento y la congestión vehicular. En este sentido, la tecnología y el Internet de las Cosas (IoT) desempeñan un papel fundamental en el desarrollo de sistemas de estacionamiento inteligentes que ayuden a los conductores a encontrar plazas disponibles de manera eficiente.

En línea con los desafíos planteados anteriormente, este proyecto propone el desarrollo e implementación de un nodo sensor de baja energía ubicado en el suelo de un estacionamiento privado. Este sensor detectará la ocupación de las plazas de estacionamiento por vehículos y se comunicará con un nodo concentrador para transmitir la información en tiempo real. Finalmente, el nodo concentrador enviará los datos a un servidor en la nube, permitiendo que los conductores encuentren plazas de estacionamiento libres a través de sus teléfonos móviles.

% SUB-SECCIÓN
\subsection{Objetivos del proyecto}
El propósito fundamental de este proyecto es facilitar la experiencia de estacionamiento de los conductores al proporcionarles una herramienta efectiva para encontrar plazas de estacionamiento disponibles en las cercanías de su destino, haciendo uso de una aplicación móvil.

Entonces, el objetivo principal se resume en:

\begin{itemize}
    \item Ayudar a los conductores a localizar y asegurar plazas de estacionamiento cercanas a su destino de manera rápida y eficiente a través de una aplicación móvil.
\end{itemize}

Para lograr este objetivo central, se establecen varios objetivos secundarios que se interrelacionan y se combinan en un enfoque integral:

\textbf{Objetivos Secundarios:}
\begin{itemize}
    \item Diseñar un sistema integral que abarque la detección de vehículos, la notificación a un servidor y el desarrollo de una aplicación móvil.
    \item Implementar un mecanismo de detección de vehículos que permita el monitoreo en tiempo real de la disponibilidad de espacios de estacionamiento.
    \item Establecer una comunicación efectiva entre los sensores de detección de vehículos y un servidor central.
    \item Desarrollar una aplicación móvil intuitiva y fácil de usar que proporcione a los conductores información en tiempo real sobre la disponibilidad de plazas de estacionamiento y les permita reservar espacios.
\end{itemize}

Estos objetivos secundarios se combinan para lograr el objetivo primario. El proyecto busca abordar desafíos comunes relacionados con la congestión de tráfico y la falta de estacionamientos, proporcionando a los usuarios una herramienta efectiva para abordar estos problemas cotidianos.


% SUB-SECCIÓN
\subsection{Alcance}
Definimos el alcance primario de este proyecto como: 
\begin{itemize}
    \item Comparar los distintos tipos de sensores utilizados para la detección de vehículos en el contexto de estacionamientos.
    \item Desarrollar un prototipo de nodo sensor capaz de detectar la disponibilidad de una plaza de estacionamiento.
    \item Realizar un estudio comparativo de las tecnologías de comunicación existentes en el área, con el fin de determinar la más adecuada para la implementación del proyecto.
    \item Desarrollar un sistema que haga uso de los datos recopilados por el nodo sensor, permitiendo a los usuarios reservar plazas de estacionamiento.
\end{itemize}

% SUB-SECCIÓN
\subsection{Limitaciones}
A continuación, se enumeran las limitaciones del proyecto, que están determinadas por el tiempo de desarrollo y los recursos disponibles. Estas limitaciones establecen las características que estarán fuera del alcance del proyecto:
\begin{itemize}
    \item Detección de vehículos que no sean un automóvil
    \item El tamaño del producto está limitado al tamaño de la plataforma elegida y sus componentes.
    \item No se desarrollará la vista para el administrador.
    \item El prototipo del nodo sensor se realizará utilizando kits de desarrollo.
    \item El diseño mecánico solo será presentado de forma teórica.
\end{itemize}

\newpage
\section{Antecedentes y trabajo relacionado}
En este trabajo final de grado, nos enfocaremos principalmente en el desarrollo de un sistema de estacionamiento inteligente que permita asistir (principalmente) a los conductores para encontrar lugares libres en un distrito de estacionamientos específicos. En conjunto con esto, un objetivo importante del sistema es ayudar a reducir el tiempo de búsqueda de un lugar, lo que se traduce en una disminución de contaminación de CO2 y también de congestión del tráfico.

En este capítulo, revisaremos el background de los \textit{Smart Parking Systems} actuales y los desafíos como así también discutiremos brevemente el trabajo relacionado.

% SUB-SECCIÓN
\subsection{Estado del arte, crítica y propuesta}
El problema de enrutamiento de vehículos (VRP) es uno de los problemas más estudiados en los sistemas de transportes inteligentes (ITS) en la actualidad. El crecimiento de las ciudades en términos de población y número de vehículos ha llevado a la búsqueda de alternativas para mejorar el sistema de transporte. La gran cantidad de vehículos intentando acceder a áreas que permanecen permanentemente congestionadas significa que encontrar un espacio para estacionar a menudo es difícil, lo que genera aún más caos en el tráfico, mayor consumo de combustible y, por lo tanto, mayores emisiones de gases de efecto invernadero.

Para aliviar dicha congestión de tráfico y mejorar la comodidad de los conductores, muchos sistemas de estacionamiento inteligentes tienen como objetivo satisfacer a las partes involucradas. Los sistemas actuales de estacionamiento inteligente o sistemas de orientación de estacionamiento tienen la siguiente clasificación: 

\begin{itemize}
    \item Búsqueda a ciegas: es una estrategia básica que los conductores emplean cuando carecen de información sobre estacionamientos disponibles. En esta modalidad, los conductores siguen buscando espacios de estacionamiento dentro de una cierta distancia a su destino. Si encuentran un espacio disponible, dejan de buscar; de lo contrario, amplían el área de búsqueda y continúan buscando espacios vacantes en los estacionamientos cercanos de forma constante.
    \item Parking Information Sharing (PIS): Este mecanismo se adopta comúnmente en el diseño actual de los sistemas inteligentes de estacionamiento \cite{ref7}. Después de que el sistema de estacionamiento inteligente publique la información de disponibilidad de estacionamiento a los conductores en cierta área, el conductor decidirá su destino de estacionamiento deseado donde el estacionamiento tenga espacios disponibles, según la información de disponibilidad de estacionamiento obtenida. Sin embargo, si el número de espacios vacantes en un estacionamiento es muy limitado en horas ocupadas, es probable que aumente el número de conductores que demandan estos espacios de estacionamiento según la información de estacionamiento. Este fenómeno se conoce como "múltiples autos persiguiendo un solo espacio", lo que puede causar una congestión grave.
    \item Buffered PIS (BPIS): Para abordar el problema del fenómeno "múltiples autos persiguiendo un solo espacio", algunos diseñadores de sistemas inteligentes de estacionamiento modifican el mecanismo PIS. Reducen intencionalmente el número de espacios vacantes al publicar la información de disponibilidad en tiempo real, para mantener un buffer. Por lo tanto, aunque puede haber más conductores persiguiendo los espacios limitados disponibles, el sistema tiene algunos espacios adicionales para evitar conflictos. Sin embargo, es difícil determinar el número de espacios en el buffer. Si el buffer es demasiado pequeño, el problema de "múltiples autos persiguiendo un solo espacio" no se eliminará. Si es demasiado grande, la utilización de los espacios de estacionamiento será baja.
\end{itemize}

Como se mencionó anteriormente, el sistema de búsqueda a ciegas es un sistema de bucle abierto en el que los usuarios toman decisiones sin considerar el estado del sistema. Las estrategias PIS y BPIS permiten a los conductores tomar decisiones basadas en el estado del sistema (por ejemplo, información de disponibilidad de estacionamiento). Sin embargo, los fenómenos de múltiples autos persiguiendo un solo espacio no se pueden eliminar por completo y muchas veces guían a los conductores a sus destinos de estacionamiento “deseados” empeorando la situación, dado que no son lo suficientemente “inteligentes”. Por lo tanto, es muy deseable proporcionar una estrategia efectiva para abordar estas preocupaciones.

En este proyecto, diseñamos e implementamos el prototipo de un Sistema de Estacionamiento Inteligente basado en Reservas  que no solo transmite información de estacionamiento en tiempo real a los conductores como parte de una aplicación, sino que también ofrece un servicio de reserva como parte de un servicio dirigido al usuario. Basado en técnicas avanzadas de sensores y comunicación móvil, este sistema procesa flujos de datos de sensores con marca de tiempo de la red del estacionamiento y publica esta información de disponibilidad en la aplicación a diseñar. Luego los conductores pueden obtener esta información desde sus smartphones, por lo tanto si un conductor hace una reserva con éxito, garantiza un espacio de estacionamiento disponible para él. Y el conductor puede estacionarse en el espacio reservado sin necesidad de buscar.
\newpage
\section{Marco Teórico}
En este capítulo se presentan conceptos relevantes que se utilizaran como investigación en este proyecto. También se incluyen presentaciones de tecnologías disponibles que pueden no formar parte de la implementación del proyecto, pero que ilustran la variedad de tecnologías alternativas y ayudan a decidir cuál es la mejor.

Primeramente, en la sección 3.1 serán presentados los conceptos de IoT y tecnologías inalámbricas disponibles. La sección 3.2 introduce edge computing. Esta sección presenta los conceptos que envuelven al edge computing y explica las tecnologías existentes. Posteriormente en la sección 3.3 se presenta el cloud computing y su conexión con edge computing. Finalmente 3.4 presenta los diferentes tipos de sensores disponibles como potenciales candidatos de este trabajo.

% SUB-SECCIÓN
\subsection{IoT (Internet of Things)}
En esta sección desarrollaremos los conceptos detrás del “Internet de las Cosas” y las diferentes tecnologías de comunicación inalámbricas que existen en este, junto también al porqué de la elección de LoRa y posterior su implementación en este proyecto.

% SUB-SECCIÓN
\subsubsection{De Internet a Internet de las Cosas}
Internet es una red de miles de millones de computadoras interconectadas, principalmente a través de cables de alta velocidad. Estas computadoras alojan una amplia variedad de servidores (servidores web, servidores de almacenamiento, servidores de correo electrónico, etc.) y clientes. La Internet está enlazada mediante una amplia gama de tecnologías electrónicas, inalámbricas y ópticas de redes. Internet transporta una vasta cantidad de recursos y servicios de información, como las aplicaciones de la World Wide Web (WWW), el correo electrónico, la telefonía y el intercambio de archivos. Por lo tanto, conectar objetos a Internet conlleva muchos beneficios.

El mundo ha experimentado estos beneficios con nuestros teléfonos celulares, computadoras portátiles y tabletas, pero esto también es cierto para todo lo demás, que esta proliferación de objetos se ha convertido en una nueva tendencia y continúa creciendo en organizaciones y empresas. Con el surgimiento de la Internet de las Cosas (IoT), se ha ampliado la conexión a objetos físicos, como sensores y actuadores, que recopilan y comparten datos a través de la Internet.

El IoT abarca una variedad de aplicaciones, como las ciudades inteligentes, donde se utilizan sensores y dispositivos conectados para monitorear y gestionar el tráfico, la iluminación, la recolección de basura y otros servicios urbanos. Estos dispositivos inteligentes recopilan información en tiempo real y la envían a estaciones base inteligentes, que analizan los datos para tomar decisiones informadas y mejorar la eficiencia de las operaciones urbanas.

Además, el mismo ha facilitado la creación de sistemas de recopilación de datos más eficientes y precisos en diversos campos, como la agricultura, la salud y la industria. Los sensores integrados en los dispositivos recolectan datos sobre el entorno, el rendimiento de las máquinas o la salud de las personas, permitiendo un monitoreo detallado y la toma de decisiones basada en datos en tiempo real.

Algo para destacar del IoT es que el mismo supone una interconexión entre estos dispositivos físicos, pero con la particularidad que generalmente utilizan tecnología inalámbrica en su interconexión y no cableada. Esto último supone todo un reto tecnológico en la búsqueda de la mejor tecnología para lograr la mayor performance de un dispositivo de acuerdo a las necesidades que el mismo deba cumplir.


% SUB-SECCIÓN
\subsubsection{IoT - Tecnologías de comunicación y sus limitaciones}
Durante las últimas décadas, las comunicaciones inalámbricas han sido objeto de mucho entusiasmo debido a su creciente integración en la vida cotidiana. Como resultado, han evolucionado significativamente desde los primeros sistemas de voz hasta las sofisticadas plataformas de comunicación integradas de hoy en día que ofrecen numerosos servicios utilizados por miles de millones de personas en todo el mundo. El Internet de las cosas (IoT) se considera la próxima revolución de las comunicaciones, que desempeñará un papel importante en la mejora de la eficiencia en la gestión de recursos humanos, naturales y energéticos, así como en la optimización de los procesos de producción. Como consecuencia, se estimó que aproximadamente 12 billones de dispositivos de IoT estaban conectados en el año 2020 [fuente] y con continua tendencia a crecer. Por otro lado, a medida que los sistemas de IoT evolucionan, nos enfrentamos a limitaciones inherentes que impiden mejoras adicionales en el rendimiento. Esto implica que sea necesario desarrollar tecnologías adecuadas que satisfagan esos requisitos (en nuestro caso, es selección adecuada dado que no desarrollaremos la tecnología de comunicación, solo la adoptaremos).

Hasta hace poco, no existía una tecnología económica, flexible y confiable para la conexión de dispositivos de IoT en la red. Las soluciones tradicionales, como las redes inalámbricas de corto alcance (por ejemplo, Bluetooth, ZigBee, Z-Wave) y las redes locales inalámbricas (WLAN, por ejemplo, WiFi, HiperLAN), así como las redes celulares (por ejemplo, GSM, LTE), aunque permiten la conexión inalámbrica de dispositivos de IoT en la red, suelen tener un alto costo, consumo de energía, complejidad y baja confiabilidad. Como resultado, se ha desarrollado recientemente la tecnología de redes de área amplia de bajo consumo de energía (LPWAN, por sus siglas en inglés). Las LPWAN se consideran excelentes candidatas para aplicaciones de IoT, ya que prometen una alta eficiencia energética, bajo consumo de energía y capacidades de cobertura amplia.

A continuación, presentaremos las diversas tecnologías de comunicación que hemos evaluado junto con sus principales características y limitaciones.

% SUB-SECCIÓN
\subsubsubsection{WiFi}
La tecnología de Wireless Fidelity (WiFi) está diseñada para conectar dispositivos electrónicos en una red inalámbrica de área local (WLAN, por sus siglas en inglés). WiFi se basa en la familia de estándares IEEE 802.11, que opera en las bandas no licenciadas de 2.4 GHz y 5 GHz disponibles en todo el mundo. WiFi incluye los estándares IEEE 802.11a/b/g para WLAN.

WiFi tiene un ancho de banda masivo de 22 MHz y, como resultado, permite alcanzar velocidades de transferencia de datos muy rápidas. La velocidad de transferencia de datos es de 54 Mbps Y puede llegar incluso a 800 Mb/s con un ancho de banda de 40 MHz. WiFi utiliza el protocolo de acceso al canal Carrier Sense Multiple Access with Collision Avoidance (CSMA/CA) y, opcionalmente, un mecanismo de solicitud de envío/confirmación de envío (RTS/CTS). Hasta el 2017, se estimaba que había más de 7 mil millones de dispositivos con tecnología WiFi en uso \cite{ref9}.

En el mundo del IoT, WiFi se utiliza para muchas aplicaciones, como monitoreo y gestión inalámbricos remotos de luces, enchufes eléctricos, vigilancia, alarmas, electrodomésticos, control climático (como control de temperatura y humedad), medición, control y diagnóstico de fabricación, equipos médicos, etc. WiFi es una tecnología clave en el desarrollo del IoT y proporciona un amplio campo para diversas soluciones de IoT.

A pesar de que WiFi es el protocolo de comunicación inalámbrica más extendido y conocido, su uso generalizado en el mundo del IoT está principalmente limitado por un mayor consumo de energía debido a la necesidad de mantener una alta intensidad de señal y una transferencia rápida de datos para una mejor conectividad y confiabilidad. La principal desventaja de WiFi es su consumo relativamente alto de energía en entornos IoT de tipo outdoor.

% SUB-SECCIÓN
\subsubsubsection{Bluetooth}
Bluetooth, también conocido como el estándar IEEE 802.15.1, es una tecnología de conectividad de corto alcance. La misma se considera una solución clave para el futuro del mercado de dispositivos electrónicos portátiles debido a su amplia integración con los teléfonos inteligentes.

Bluetooth se basa en un sistema de radio inalámbrico diseñado para dispositivos de corto alcance y económicos que reemplazan los cables de los periféricos de computadora, como ratones, teclados, impresoras, etc. Esta gama de periféricos se utiliza en un tipo de red conocida como red inalámbrica de área personal (WPAN, por sus siglas en inglés). Diseñado con eficiencia en costos y consumo reducido de energía, el protocolo Bluetooth de baja energía (BLE, por sus siglas en inglés) requiere muy poca energía del dispositivo. Sin embargo, esto implica un compromiso: cuando se transfieren frecuentemente grandes cantidades de datos, BLE no es una solución efectiva, ya que se consume más energía. La velocidad de transferencia de datos es de 3 Mb/s. Bluetooth opera en la banda de frecuencia de radio de 2.4 GHz ISM que está disponible para uso sin licencia en todo el mundo.

Un conjunto de dispositivos Bluetooth que comparten un canal común se denomina piconet. Un piconet es una configuración en forma de estrella en la que el dispositivo en el centro desempeña el papel de maestro y todos los demás dispositivos funcionan como esclavos. Hasta siete esclavos pueden estar activos y ser atendidos simultáneamente por el maestro. Si el maestro necesita comunicarse con más de siete dispositivos, puede hacerlo pidiendo a los dispositivos esclavos activos que pasen al modo de bajo consumo de energía y luego invitando a otros esclavos estacionarios a activarse en el piconet. Este comportamiento puede repetirse, lo que permite que un maestro sirva a un gran número de esclavos.

Sin embargo aunque este protocolo parece prometedor para nuestra solución, tiene algunos inconvenientes en la aplicación del mismo, principalmente relacionados con el consumo de energía, donde el BLE (Bluetooth Low Energy) ha mejorado su performance, todavía sigue siendo mayor en consumo que otras alternativas como Zigbee o LoRaWAN. Por lo tanto podríamos decir que en aplicaciones donde nos interese tener un tiempo de vida prolongado, esto sería un problema. Sumado además a lo anterior, el mismo en áreas metropolitanas congestionadas por otras tecnologías como WiFi, dado que opera en una banda de 2.4 GHz puede verse afectado por interferencias y/o congestión de la señal, lo cual puede afectar la calidad y estabilidad de la conexión.

% SUB-SECCIÓN
\subsubsubsection{IEEE 802.15.4 / ZigBee}
ZigBee es un estándar de comunicación basado en IEEE 802.15.4 que se utiliza ampliamente en aplicaciones de redes inalámbricas de malla. Utiliza un enfoque de autoorganización y autorreparación, lo que significa que la red puede adaptarse y corregir automáticamente problemas de conectividad. Se destaca por su bajo consumo de energía y su capacidad para proporcionar conectividad confiable en diversos entornos. Este se centra en aplicaciones de baja potencia y ofrece soporte para servicios de red, seguridad y aplicaciones. Se utiliza en una variedad de sectores, incluidos sistemas de gestión de tráfico, electrónica doméstica e industria de máquinas.

Podemos considerar que estas son las características principales de esta tecnología:
\begin{itemize}
    \item Bajo consumo de energía, con una vida útil de la batería que varía desde meses hasta años.
    \item Tres bandas sin licencia: 2.4-2.4835 GHz, 902-928 MHz y 868-870 MHz. El número de canales asignados a cada banda de frecuencia es fijo en dieciséis, diez y uno respectivamente. La banda de frecuencia más alta se puede utilizar en todo el mundo y las dos bandas más bajas en las áreas de América del Norte y Europa
    \item Las tasas máximas de datos permitidas para cada una de estas bandas de frecuencia son de 250 kbps a 2.4 GHz, 40 kbps a 915 MHz y 20 kbps a 868 MHz.
    \item Acceso al canal mediante Carrier Sense Multiple Access with Collision Avoidance (CSMA/CA) para acceder al medio compartido.
\end{itemize}

Más allá de las ventajas que ofrece esta tecnología, también debemos mencionar que tenemos asociados problemas de complejidad de configuraciones en la administración de una red ZigBee con respecto a otros competidores como LoRa y también su precio en el mercado argentino es superior.

% SUB-SECCIÓN
\subsubsubsection{Sigfox}
Sigfox es una tecnología LPWAN diseñada por una empresa francesa que ofrece una solución de conectividad IoT de extremo a extremo basada en su propia tecnología patentada.

Sigfox despliega sus propias estaciones equipadas con radios y las conecta a los servidores de backend mediante un red basada en IP. Luego los dispositivos finales se conectan a estas estaciones base utilizando la banda ISM ultra estrecha (ultra-narrow band). Esta tecnología utiliza bandas ISM sin licencia y al emplear ultra-narrow band logra optimizar el ancho de banda de frecuencias de forma eficiente, haciendo que los niveles de ruido sean muy bajos, lo que permite que el consumo de energía sea muy bajo a su vez, la sensibilidad del receptor sea alta.

Inicialmente esta tecnología solo admitía la comunicación uplink, aunque evolucionó posteriormente a una tecnología bidireccional con una importante asimetría de enlace. Haciendo que la comunicación downlink, es decir, los datos que se envían desde el backend hacia los dispositivos finales solo es posible si se ha producido en primer lugar una comunicación uplink.

Si bien Sigfox ha mejorado a lo largo de los años, todavía cuenta con una limitación en cuenta a la cantidad de mensajes diarios que pueden ser enviados por día desde un mismo dispositivo final, el cual está limitado a 140 mensajes diarios con una carga útil de 12 bytes y cuando analizamos el downlink (los que un dispositivo puede recibir) esto es aún peor con una limitación de 4 mensajes diarios. Al margen de lo mencionado anteriormente una gran característica que tiene esta tecnología es la fiabilidad de la comunicación mediante la diversidad de tiempo y frecuencia así como la duplicación de transmisión. Cada mensaje se transmite varias veces
(tres por defecto) por diferentes canales de frecuencia. Como las estaciones base pueden recibir mensajes simultáneamente en todos los canales, el dispositivo final puede elegir aleatoriamente un canal de frecuencia para transmitir sus mensajes. Esto simplifica el diseño del dispositivo final y reduce su coste.

Cabe aclarar también que aunque Sigfox presente muchas buenas características que la harían un candidato ideal, además de las limitaciones anteriormente mencionadas en cuanto a la cantidad de mensajes, también tenemos el apartado de cobertura de la red. Dado que esta tecnología depende de torres dedicadas, que deben ser instaladas por el propio fabricante, esto pasa a ser un problema mayor para poder utilizarla dado que la implementación de la misma no tiene cobertura total en el país de acuerdo con el sitio oficial de Sigfox. Lo que hace que la misma sea rechazada para una posterior elección.

% SUB-SECCIÓN
\subsubsubsection{Tecnología LoRa}
LoRa es un esquema de modulación utilizado para la comunicación de largo alcance y es un protocolo propietario desarrollado por Semtech. Debido a que es propietario, la información sobre su implementación no está fácilmente disponible. Sin embargo, se ha publicado parte de la información por parte de Semtech y también el protocolo ha sido analizado e invertido por los investigadores hasta un punto en el que su implementación se ha comprendido bien.

LoRa está basado en la técnica de modulación CSS (Chirp Spread Spectrum) y puede utilizar uno o más canales. Este se ha convertido en una de las tecnologías preferidas para las redes de IoT gracias a su capacidad para proporcionar largo alcance y bajo consumo de energía. Opera en la banda no licenciada ISM y utiliza bandas de frecuencia sub-GHz. También utiliza seis factores de propagación para adaptar la velocidad de datos y el alcance; donde un mayor factor de dispersión permite un mayor alcance a costa de una menor velocidad de datos, y viceversa. La velocidad de datos de LoRa oscila entre 300 bps y 50 kbps, dependiendo del factor  de dispersión y del ancho de banda del canal. Finalmente algo para notar en esta tecnología es que la misma tiene un parámetro importante llamado SF (Spreading Factor) que es ortogonal, lo que habilita a múltiples señales a ser transmitidas por el mismo canal concurrentemente sin interferencia y luego en el receptor se pueden detectar los diferentes paquetes enviados con los diferentes SF. La única condición aquí es que dos SF no deben ser iguales. Esto es algo importante para notar dado que significa un aumento importante en la eficiencia y el ancho de banda del canal.

% SUB-SECCIÓN
\subsubsubsection{LoRaWAN}
LoRaWAN es el protocolo de comunicación a nivel de enlace (OSI 2) y red (OSI 3) que va sobre la capa física LoRa, el protocolo de comunicación LoRaWAN es abierto, esto permite que multitud de fabricantes puedan desarrollar dispositivos y de esta forma abaratar los costes de estos. De forma análoga a una red Ethernet se podría decir que LoRa son los cables que conectan los dispositivos en una red Ethernet y LoraWAN es la comunicación de los dispositivos en la dirección MAC y de la dirección IP de red de los dispositivos en la red Ethernet.

El protocolo y la arquitectura de red tienen la mayor influencia en determinar la vida útil de la batería de un nodo, la capacidad de la red, la calidad del servicio, la seguridad y la variedad de aplicaciones atendidas por la red.

Una gran ventaja de este protocolo es que los nodos no están asociados con una gateway específica. En su lugar, los datos transmitidos por un nodo suelen ser recibidos por múltiples gateways, y cada uno retransmitirá el paquete recibido desde el nodo final al servidor de red basado en la nube a través de algún enlace de retorno (ya sea celular, Ethernet, satélite o Wi-Fi). En esta tecnología, la inteligencia y complejidad se transfieren al servidor de red, que gestiona la misma y filtra los paquetes redundantes recibidos, realiza verificaciones de seguridad, programa los acuses de recibo a través de la puerta de enlace óptima y realiza adaptaciones de la velocidad de datos, entre otras funciones. Entonces, por ejemplo si un nodo es móvil o se está moviendo, no se necesita una transferencia de conexión de un gateway a otro, lo cual es una característica importante para habilitar aplicaciones de seguimiento de activos.

LoRaWAN ofrece una ventaja energética significativa, ya que los nodos se comunican de forma asíncrona, transmitiendo datos solo cuando están listos, ya sea por eventos o programación. Esto contrasta con redes en malla o sincrónicas, como las celulares, donde los nodos deben "despertar" frecuentemente para sincronizarse y verificar mensajes, lo que resulta en un consumo de energía más alto y una menor vida útil de la batería.

Como parte de la arquitectura de este protocolo también se definen tipos de nodo, características de seguridad, etc. Estas características permiten que una red LoRaWAN tenga una capacidad muy alta de escalabilidad muy alta. Y que se pueda implementar la misma con una cantidad mínima de infraestructura y a medida que se necesitan más capacidad se agreguen gateways para manejarlas. Por estas características mencionadas anteriormente, creemos que esta es una tecnología ideal para nuestro proyecto.

% SUB-SECCIÓN
\subsection{Edge Computing}
El Edge Computing, o computación de borde en español, es un paradigma que persigue el objetivo de acercar el almacenamiento y procesamiento de datos cerca del lugar donde se producen. Este paradigma tiene como objetivo lograr una menor latencia, mayor ancho de banda disponible y más privacidad.

La computación de borde, presenta una respuesta al desafío de procesar la gran cantidad de datos generados por dispositivos electrónicos y sensores de manera oportuna. De acuerdo a una predicción de Cisco, para el año 2023 se espera que el 70 por ciento de la población mundial tenga conectividad móvil, es aquí donde nos preguntamos si ¿la computación de borde ayudará a enfrentar la creciente necesidad de procesamiento de datos?.

Como podemos analizar a nuestro alrededor, cada día son más los dispositivos IoT que se conectan a la internet, como luces LED, cámaras, smartwatches, redes de sensores, etc. Y estos comienzan a tener implicancia cada vez mayor en la toma de decisiones de áreas como atención médica, transporte, seguridad, etc. Entonces, con este paradigma se busca aprovechar las capacidades existentes en procesamientos de datos en sistemas integrados para lograr una menor congestión en los sistemas distribuidos, como así también mejorar la velocidad de actuación de los nodos finales.

Entonces ¿qué es un nodo de borde? A grandes rasgos podemos decir que cualquier componente que pueda realizar operaciones de cálculo o almacenar información acerca de la localización donde los datos fueron generados cumple con la definición. En comparación a otros paradigmas donde generalmente se recolecta la información en el edge, luego se la transfiere a través de la red a un sistema de servidores densos y luego de ser procesada en el cloud la misma vuelve a al edge. Siendo que los poderes de cómputo son mayores hoy en día en los dispositivos de borde, se elimina esta necesidad de transferir toda la información para ser procesada.

Para este proyecto, como dispositivo de borde emplearemos una Raspberry Pi, que nos ayuda junto con una red de ML a detectar si efectivamente debemos informarle al servidor sobre la disponibilidad/ocupación de una plaza de estacionamiento o no. Profundizaremos esto más adelante, pero a grandes rasgos los nodos sensores transmiten la información mediante LoRa a nuestro dispositivo de borde, y una vez que este tenga la información procesada, le informará al servidor sobre el estado de la plaza logrando reflejar la reserva o disponibilidad de la misma.

De esta forma, evitamos que los sensores congestionen la red, con el envío de datos al servidor, lo que en gran medida implica un costo elevado del servicio.

% SUB-SECCIÓN
\subsection{Cloud Computing}
La computación en la nube está evolucionando como un nuevo modelo de computación diseñado para ofrecer entornos informáticos dinámicos y bajo demanda para los usuarios, brindando un servicio rápido, seguro y personalizable. Esta tecnología evita que las empresas tengan que encargarse de aprovisionar, configurar o gestionar los recursos y permite que paguen únicamente por los que usen.

% SUB-SECCIÓN
\subsubsection{Tipos de modelo}
Hay tres tipos de modelos de servicio de cloud computing:

\begin{itemize}
    \item Infraestructura como servicio (IaaS)
    \item Plataforma como servicio (PaaS)
    \item Software como servicio (SaaS)
\end{itemize}

\subsubsubsection{Infraestructura como servicio (IaaS):}
La infraestructura como servicio proporciona a las empresas recursos informáticos, incluyendo servidores, redes, almacenamiento y espacio en centro de datos con pago en función del uso. Las ventajas del IaaS son:

\begin{itemize}
    \item No es necesario invertir en tener hardware propio
    \item La infraestructura escala ondemand para dar soporte a las cargas de trabajo dinámicas.
    \item Se pueden crear servicios flexibles e innovadores a demanda.
\end{itemize}

\subsubsubsection{Plataforma como servicio (PaaS):}
La plataforma como servicio proporciona un entorno basado en cloud con todos los requisitos necesarios para dar soporte a todo el ciclo de vida de creación y puesta en marcha de aplicaciones basadas en web (cloud), sin el coste y la complejidad de comprar y gestionar el hardware, software, aprovisionamiento y alojamiento necesario. Las ventajas de este tipo de modelo son:

\begin{itemize}
    \item Acelerar el desarrollo y la comercialización de aplicaciones.
    \item Desplegar en cuestión de minutos nuevas aplicaciones web en cloud.
    \item Reducir la complejidad con middleware como servicio.
\end{itemize}

\subsubsubsection{Software como servicio (SaaS):}
Las aplicaciones basadas en cloud, o software como servicio, se ejecutan en sistemas distantes "en el cloud", que pertenecen y son administrados por otros y que están conectados a los sistemas de usuario a través de Internet y, por lo general, de un navegador web. Algunas ventajas son:

\begin{itemize}
    \item Las aplicaciones y los datos son accesibles desde cualquier sistema conectado.
    \item No se pierden datos si su sistema falla, ya que los datos están en el cloud.
    \item El servicio permite escalar dinámicamente en función de las necesidades de uso.
\end{itemize}

\subsubsection{¿Cómo funciona el Cloud computing?}
Los modelos de servicio de cloud computing se basan en el concepto de compartir recursos informáticos, software e información bajo demanda por Internet. Las empresas o personas físicas pagan para acceder a un grupo virtual de recursos compartidos, incluidos servicios de computación, almacenamiento y redes, que se encuentran en servidores remotos propiedad de proveedores de servicios y gestionados por ellos.

Una de las numerosas ventajas que proporciona el cloud computing es el modelo de pago por uso, lo que permite a las organizaciones escalar de manera rápida y eficiente sin la obligación de adquirir y mantener sus propios centros de datos físicos y servidores.

Es decir, el cloud computing utiliza una red (normalmente, Internet) para conectar a los usuarios a una plataforma en la nube donde solicitan y acceden a servicios informáticos alquilados. Un servidor central gestiona toda la comunicación entre los dispositivos y los servidores del cliente para facilitar el intercambio de datos. Las funciones de seguridad y privacidad son componentes habituales para mantener la seguridad de esta información.

A la hora de adoptar una arquitectura de cloud computing, no hay una solución universal. Puede que lo que funcione para un proyecto (o empresa) no se adapte a otro proyecto dadas las necesidades del mismo. Es de hecho, que esta flexibilidad y versatilidad son dos de los aspectos distintivos de la nube que permiten a las empresas adaptarse rápidamente a los cambios de los mercados.

Hay tres modelos diferentes de despliegue de cloud computing: nube pública, nube privada y nube híbrida.
\begin{itemize}
    \item Los clouds públicos pertenecen y son administrados por empresas que ofrecen a través de una red pública acceso rápido a recursos informáticos asequibles. Con los servicios de cloud público, los usuarios no necesitan adquirir hardware, ni software ni infraestructura de soporte, ya que pertenece a los proveedores y lo gestionan ellos.
    \item Un cloud privado se trata de una infraestructura que utiliza únicamente una única organización, ya sea gestionada internamente o por terceros y alojada internamente o externamente. Los clouds privados aprovechan la eficiencia del cloud, a la vez que ofrecen un mayor control de los recursos y evitan la multitenencia.
    \item Un cloud híbrido utiliza una base de cloud privada, combinada con la integración estratégica y el uso de servicios cloud públicos. En realidad, un cloud privado no puede existir aislado del resto de los recursos de TI de una empresa ni del cloud público. La mayoría de las empresas con clouds privados evolucionarán para gestionar cargas de trabajo en todos los centros de datos, clouds privados y clouds públicos, creando así clouds híbridos.
\end{itemize}


\newpage
\section{Análisis de la solución}
En este capítulo presentaremos una visión general de la implementación del sistema de reservas propuesto, como implementamos la detección del vehículo en un slot del estacionamiento y la integración general entre la aplicación desarrollada, el backend, nodo sensor y de borde. También se presenta la arquitectura del mismo, describiendo el papel de cada componente que constituye el sistema.


% SUB-SECCIÓN
\subsection{Introducción}
Uno de los objetivos principales en este proyecto final, es el desarrollo de un nodo que sea capaz de detectar la ocupación de una plaza de estacionamiento y el desarrollo de un sistema integral de reservas. En resumen se utilizaron las ventajas ofrecidas por el concepto de arquitectura de sistemas IoT (Internet of Things), el paradigma de Edge Computing (Computación de Borde) y el poder de la nube (Cloud Computing) como entorno de ejecución. A continuación se deja una representación simple de la arquitectura de la misma.

\insertimage[\label{architecture_minimal_diagram}]{secciones/section_4/images/minimal_architecture}{width=17cm}{Arquitectura minimalista del sistema}

Y en la imagen a continuación se presenta como sería el caso de uso real:
\insertimage[\label{representacion_estacionamiento_real}]{secciones/section_4/images/parking_lot}{width=17cm}{Representación Estacionamiento con sensores}

% SUB-SECCIÓN
\subsection{Arquitectura del sistema}
Haciendo referencia a lo mencionado en el apartado anterior, haremos el desarrollo de los siguientes componentes del sistema:

\begin{itemize}
    \item Nodo sensor
    \item Nodo de borde (collector)
    \item Software integral (arquitectura Cloud-based)
\end{itemize}

En resumen, el sistema realiza la interconexión y comunicación entre los componentes físicos del mismo (sensores) y los componentes computacionales, que logran conectar al usuario final con el de una plaza de estacionamiento.

% Insertar Nodo Sensor
\subsubsection{Nodo sensor}
El nodo sensor de nuestro proyecto está compuesto de las siguientes etapas:

\begin{itemize}
    \item Etapa de adquisición de datos
    \item Etapa de procesamiento
    \item Etapa de comunicación
\end{itemize}

Como se puede ver en la figura a continuación:

\insertimage[\label{architecture_nodo_sensor}]{secciones/section_4/images/sensor}{width=18cm}{Arquitectura del nodo sensor}

Al momento de la elección de un microcontrolador para llevar a cabo nuestro prototipo, nos enfocamos en la etapa de comunicaciones principalmente. De capítulos anteriores hemos optado por LoRa como la tecnología a utilizar en este proyecto, por lo que en nuestra búsqueda, encontramos que el hardware que presentaba la mejor relación costo-beneficio para un prototipo eran las placas WiFi LoRa 32 V2 de la compañía HELTEC.

El equipo Heltec WiFi LoRa 32 es un dispositivo orientado a IoT, diseñado y producido por Heltec Automation. La tarjeta de desarrollo basa su funcionamiento en el microcontrolador ESP32, y se vale del chip integrado SX1278, para las funciones de comunicación LoRa, estos dos dispositivos interactúan por medio de una interfaz Serial Peripheral Interface (SPI).

\insertimage[\label{heltec_lora}]{secciones/section_4/images/heltec}{width=13cm}{Placa de desarrollo Heltec Lora WiFi v2}

Luego de la elección del kit de desarrollo, nos enfocamos en la determinación del sensor, y como hemos visto de la sección de “ensayos y conclusiones”, optamos por utilizar dos sensores en conjunto, un magnetómetro y otro de tipo radar. A raíz de lo disponible en el mercado, concluimos que los sensores que mejor se ajustan a nuestro proyecto son: HMC5883L (magnetómetro) y VL53L0X (radar).

De esta forma queda determinado cuál va a ser el hardware de nuestro proyecto.


\subsubsubsection{Hardware}
A continuación se muestran las características técnicas de la placa WiFi LoRa 32 V2 utilizada en este proyecto, junto con los parámetros principales de los sensores MC5883L y VL53L0X.

\enabletablerowcolor[2] % Activa el color de celda
\begin{table}[ht]
    \centering
    \caption{Tabla de parámetros Heltect WiFi LoRa 32 v2}
    \begin{tabular}{|p{2cm}|*{3}{>{\raggedright\arraybackslash}p{14cm}|}}
        \hline
        \textbf{Parámetro} & \textbf{Descripción} \\
        \hline
        Master Chip & ESP32 (240MHz Tensilica LX6 dual-core+1 ULP, 600 DMIPS) \\
        LoRa Chipset & SX1276/SX1278 \\
        Wi-Fi & 802.11 b/g/n (802.11n up to 150 Mbps) \\
        Bluetooth &Bluetooth V4.2 BR/EDR and Bluetooth LE specification \\
        Hardware Resource & UART x 3; SPI x 2; I2C x 2; I2S x 1; 12-bits ADC input x 18; 8\-bits DAC output x 2; GPIO x 22; GPI x 6 \\
        Memory &8MB(64M-bits) SPI FLASH; 520 KB internal SRAM \\
        Interface &Micro USB x 1; LoRa Antenna interface(IPEX) x 1 \\
        Dimensions &51 x 25.5 x 10.6 mm \\
        \hline
        \end{tabular}
    \label{tab:tabla_parametros_heltec}
\end{table}
\disabletablerowcolor % Desactiva el color de celda

\enabletablerowcolor[2] % Activa el color de celda
\begin{table}[ht]
    \centering
    \caption{Tabla de parámetros sensor magnético HMC5883L}
    \begin{tabular}{|p{6cm}|*{2}{>{\raggedright\arraybackslash}p{5cm}|}}
        \hline
        \textbf{Parámetro} & \textbf{HMC5883L} \\
        \hline
            Voltage Supply (Vs) &2V16 - 3V6 \\
            Digital Supply (VDDIO) (max) & 1.71V - 3V7 \\
            Abs. Max VDD/VDD IO & -0.3V - 4.8V \\
            Interface &I2C \\
            I2C Address (R,W) [RW] &0x3D, 0x3C \\
            I2C rates (kHz) & 100, 400, 3400 \\
            Resolution (ADC) & 12  bits \\
            Max Gauss (survival) & Not specified. \\
            Gauss Resolution & ±2mG - ±8G \\
            Acquisition time &6ms \\
            Active current (7Hz,10Hz) &100uA \\
            Active current &Not specified. \\
            Peak Active current &Not Specified[3] \\
            Standby mode (leakage) & 2uA \\
            Operating temperature & -30°C - 85°C \\
        \hline
        \end{tabular}
    \label{tab:tabla_parametros_sensor_magnetico}
\end{table}

\enabletablerowcolor[2] % Activa el color de celda
\begin{table}[H]
    \centering
    \caption{Tabla de parámetros sensor laser VL53L0X}
    \begin{tabular}{|p{6cm}|*{2}{>{\raggedright\arraybackslash}p{5cm}|}}
        \hline
        \textbf{Parámetro} & \textbf{VL53L0X} \\
        \hline
            Ranging chip &VL53L0X \\
            Measuring distance & 2M (Max) \\
            Measuring Mode & Default, High precision, Long Distance, High Speed \\
            Infrared emission mechanish &940nm \\
            FOV & 25º \\
            Operating Voltage & 3 - 5V \\
            Operating temperature & -20°C - 80°C \\
        \hline
        \end{tabular}
    \label{tab:tabla_parametros_sensor_laser}
\end{table}


\subsubsubsection{PinOut}
A continuación se visualiza el pinout del kit de desarrollo. Más adelante incluiremos un esquemático de la conexión entre el sensor y la misma.

\insertimage[\label{heltec_pinout}]{secciones/section_4/images/pinout_heltec}{width=16cm}{Pinout placa de desarrollo Heltec Lora WiFi v2}

\subsubsubsection{Conexión}
A continuación se visualiza la conexión entre el kit de desarrollo y los sensores utilizados en este proyecto. Cabe aclarar que este es el modelo esquemático 
esta diseñado unicamente con los kits de desarrollo, más adelante incluiremos un apartado del diseño del PCB.

%\insertimage[\label{heltec_conexion_sensores}]{secciones/section_4/images/heltec_conexion_sensores}{width=13cm}{Diagrama de conexión entre Heltec y Sensores}

\insertimage[\label{heltec_diagrama_conexion}]{secciones/section_4/images/heltec_diagrama_conexion}{width=16cm}{Diagrama de conexión esquematico entre Heltec y Sensores}

\subsubsubsection{Configuración del nodo sensor}
Para poder explicar mejor el software implementado en el sensor, presentamos primero un pequeño diagrama inicial de eventos del sistema. Cabe aclarar que este es un diagrama simplificado y no incluye toda la interacción del sistema.

\insertimage[\label{diagrama_de_eventos_del_sistem}]{secciones/section_4/images/diagrama_de_eventos_del_sistem}{width=13cm}{Diagrama de eventos del sistema}

Los datos a transmitirse desde el nodo sensor hacia el colector son recopilados por la tarjeta Heltec WiFi LoRa 32. Esta tarjeta permite la interacción con el magnetómetro y el sensor de tiempo de vuelo a través del protocolo I2C.

El microprocesador ESP32 de la tarjeta Heltec, maneja el chip SX1278 (LoRa) y sus parámetros de transmisión (frecuencia, potencia, SF) así como las claves de seguridad de la red. La programación de este controlador se encuentran disponibles en el repositorio de GitHub del proyecto. Ver anexo ~\ref{codigo_sensor}

\insertimage[\label{architectura_nodo_facil}]{secciones/section_4/images/architectura_nodo_facil}{width=13cm}{Diagrama de bloques nodo sensor}

\subsubsubsection{Estimación tiempo de vida util - bateria del nodo sensor}{\label{sec:bateria_del_nodo_sensor}}
Dada la naturaleza de nuestro proyecto, uno de los aspectos clave del mismo en la implementación del nodo sensor es el diseño de este para que consuma muy poca energía y además acompañar esta implementación junto con la estimación en el tiempo de vida útil del nodo.
Al lograr estimar la vida útil de la batería, se pueden tomar mejores decisiones para realizar una planificación del mantenimiento y la sustitución anticipada de la batería, lo que contribuye a evitar interrupciones en el funcionamiento del sistema.

A continuación introduciremos los conceptos claves junto con los cálculos necesarios para lograr la implementación y la estimación del tiempo de vida.

\subsubsubsectionanumnoi{ESP8266 - Low-Power Management}
Conforme se destacó en la introducción de esta sección, resulta crucial considerar el consumo de energía durante el desarrollo del nodo sensor. Nuestra investigación revela que el microcontrolador a emplear ofrece cuatro (4) perfiles de energía distintos, como se muestra en la imagen adjunta.

\insertimage[\label{esp_low_power}]{secciones/section_4/images/esp_low_power}{width=17cm}{Modos de consumo de energía ESP8266}

Durante el modo de suspensión profunda (deep sleep), la CPU principal se apaga, mientras que el Coprocesador Ultra-Bajo Consumo (ULP) puede tomar lecturas de sensores y despertar a la CPU según sea necesario. Este tipo de perfil de energía es útil para diseñar aplicaciones en las que la CPU debe despertarse por un evento externo, un temporizador o una combinación de estos eventos, manteniendo un consumo mínimo de energía.

Debido a que la memoria RTC se mantiene activa, su contenido se conserva incluso durante el modo de suspensión profunda y se puede recuperar una vez que el chip se despierta. Para una mejor ejemplificación adjuntamos el diagrama de bloques con aquellos bloques que siguen activos durante el modo de funcionamiento Deep-Sleep.

\insertimage[\label{diagrama_low_power_bloques}]{secciones/section_4/images/diagrama_low_power_bloques}{width=15cm}{Bloques activos en modo Deep-Sleep}

\subsubsubsectionanumnoi{Funcionamiento del nodo (considerando los modos de energía)}
En apartados anteriores, hemos explicado el principio de funcionamiento de nuestro nodo sensor y los distintos modos de consumo de energía que ofrece el microcontrolador ESP8266. Ahora, detallaremos cómo funciona nuestro nodo en conjunto con los periféricos y el software diseñado.

Antes de adentrarnos, asumiremos que el nodo ya está configurado, es decir, las variables necesarias para una comunicación exitosa con la etapa de comunicaciones LoRa han sido establecidas (frecuencia, canales de TX y RX, SF, etc.). Además, los pines para la comunicación mediante I2C han sido configurados previamente. En esta ocasión, nos centraremos en el funcionamiento del nodo para llevar a cabo la detección de vehículos.

\insertimage[\label{power_modes_system}]{secciones/section_4/images/power_modes_system}{width=15cm}{Esquema simplificado de los modos de energia y su interacción.}

Tras la configuración previa, el microcontrolador entra en modo de Deep-Sleep. Durante este tiempo, toma lecturas del magnetómetro HMC5883L, procesa los datos y determina si se ha superado el valor límite establecido. Es importante mencionar que durante esta fase, el microcontrolador utiliza el coprocesador ULP para el cómputo de los datos, lo que permite reducir el consumo de energía a tan solo 10 μA durante los 500 mS que el procesador principal se encuentra apagado, según las especificaciones oficiales.

\insertimage[\label{esquema_energia}]{secciones/section_4/images/esquema_energia}{width=15cm}{Esquema de energía interno ESP8266}

Cuando se detecta que el valor límite ha sido superado, el coprocesador llama a una función previamente definida, indicando a la ESP que debe salir del modo Deep-Sleep y entrar en modo de funcionamiento activo.

En el modo activo, el microcontrolador solicita la lectura del sensor láser VL53I0x y espera una interrupción que se produce cuando la lectura está lista. Una vez que se obtiene la lectura, se compara con la última registrada para determinar si ha habido algún cambio significativo. En caso afirmativo, se envía el valor a través de LoRa.
Mientras se espera recibir una señal de ACK (acknowledgement) por parte del nodo de borde, el microcontrolador entra en un estado de sueño intermitente durante un segundo para evitar un gasto innecesario de energía.

Finalmente, una vez que se recibe la señal de ACK, el microcontrolador vuelve a ingresar en el modo de Deep-Sleep, preparado para realizar nuevas detecciones.

\subsubsubsectionanumnoi{Cálculo teórico - tiempo de vida batería}
Para poder estimar de manera precisa el tiempo de vida de la batería que se requiere para nuestro equipo, se ha realizado una cuidadosa identificación de dos patrones de consumo claramente definidos, a los cuales hemos denominado \textit{Alto Consumo} y \textit{Bajo Consumo}, cada uno con sus respectivos periodos de funcionamiento "ON" y de inactividad "OFF". Este desglose minucioso nos proporcionará una estimación más detallada y precisa del tiempo de vida esperado para la batería, permitiéndonos tomar decisiones informadas acerca de la capacidad y durabilidad óptimas para el sistema electrónico en cuestión. Este análisis está basado en que muchos de los sistemas IoT pasan una pequeña parte de su tiempo en un modo activo (o de funcionamiento) y el resto en modo de inactividad.

Para comenzar con el cálculo teórico, vamos a tomar los niveles de consumo de acuerdo a los datasheet de los fabricantes. En el sistema de bajo consumo tenemos los siguientes parámetros:

\insertindexequation[\label{consumo_ulp_modo_activo}]{I_{ULP A} = 0.0004 mA}{consumo ULP en modo activo}
\insertindexequation[\label{consumo_ulp_modo_off}]{I_{ULP Off} = 0.00001 mA}{consumo ULP en modo off}
\insertindexequation[\label{tiempo_ulp_modo_activo}]{t_{ULP A} = 2 ms}{tiempo en modo activo}
\insertindexequation[\label{tiempo_ulp_modo_off}]{t_{ULP Off} = 500 ms}{tiempo en modo off}

Por lo tanto la corriente total queda definida como:
\insertindexequation[\label{corriente_total}]{I_{ULP} = \frac{(I_{ULP A}* t_{ULP A} + I_{ULP Off} * t_{ULP Off})}{t_{ULP A} + t_{ULP Off}}}{Corriente total}{}

\insertindexequation[\label{corriente_magnetometro}]{I_{Magnetometer}= 0.00002 mA}{Corriente total}{}

Finalmente el valor de la corriente de bajo consumo queda definida como la suma del procesador ULP y la corriente consumida por el magnetómetro.

\insertindexequation[\label{corriente_I_low_power}]{I_{Low Power}= I_{ULP} + I_{Magnetometer}}{Corriente total del sistema bajo consumo}{}

En el sistema de alto consumo consideramos los siguientes valores:

\insertindexequation[\label{corriente_I_low_power}]{t_a}{medido en ms}{representa el tiempo que permanece cada auto en un slot}
\insertindexequation[\label{corriente_esp_modo_activo}]{I_{ESP A} = 25 mA}{consumo ESP en modo activo}{}
\insertindexequation[\label{corriente_ulp_modo_off}]{I_{Laser ON} = 19 mA}{consumo ULP en modo off}{}
\insertindexequation[\label{tiempo_esp_modo_activo}]{t_{ESP A} = 300 ms}{tiempo que la ESP permanece en modo activo}{}
\insertindexequation[\label{tiempo_laser_activo}]{t_{Laser On} = 40 ms}{tiempo que el laser está activo}{}

Con las variables anteriores queda definido el tiempo en la que el nodo sensor se encuentra recolectando la información sobre la ocupación de un slot de estacionamiento.

\insertindexequation[\label{corriente_esp_collecting}]{I_{ESP Collecting} = \frac{I_{ESP A} * t_{ESP A}}{t_{ESP A} + t_a}}{corriente de la ESP mientras collecta información}{}
\insertindexequation[\label{tiempo_laser_activo}]{I_{Laser Collecting} = \frac{I_{Laser ON} * t_{Laser ON}}{t_{Laser ON} + t_a}}{corriente total mientras el laser collecta información}{}
\insertindexequation[\label{tiempo_laser_activo}]{I_{Collecting} = I_{ESP Collecting} + I_{Laser Collecting} }{corriente total consumida mientras se collecta información}{}

Por último, una vez recolectada la información, tenemos la etapa de envío y recepción de información.

\insertequation{I_{ESP Light Sleep} =0.8 mA}
\insertequation{I_{sx stby}=0.0000015 mA}
\insertequation{I_{sx tx} = 100 mA}
\insertequation{I_{sx rx} = 12 mA}

\insertequation{t_{sx tx} = 175 ms}
\insertequation{t_{sx rx} = 1000 ms}
\insertequation{t_{sx stby} = 27 ms}
\insertequation{t_{ESP A} = t_{sx stby}}
\insertequation{t_{ESP_Light_Sleep} = t_{sx tx} + t_{sx rx}}

La etapa de comunicación queda definida entonces por la suma entre la corriente que consume el nodo mientras transmite y mientras recibe como se puede ver en las fórmulas a continuación.

\insertequation{I_{Transmitting\ ESP} = \frac{I_{ESP\ A} * t_{ESP\ A} + I_{ESP\ Light\ Sleep} * t_{ESP\ Light\ Sleep}}{t_a+ t_{ESP\ A}+ t_{ESP\ Light\ Sleep}}}

\insertequation{I_{Transmitting\ SX} = \frac{t_a *I_{sx\ stby} + t_{sx\ stby} * I_{sx\ stby} +  t_{sx\ rx} * I_{sx\ rx} + t_{sx\ tx} * I_{sx\ tx}}{t_a+ t_{sx\ stby} +  t_{sx\ rx} + t_{sx\ tx}}}

\insertequation{I_{Transmitting} = I_{Transmitting\ ESP} + I_{Transmitting\ SX}}
\insertequation{I_{High\ Power} = I_{Collecting} + I_{Transmitting}}

En este punto ya tenemos todos los valores de intensidad expresados y podemos estimar el tiempo de vida útil de una batería. Asumiendo que tenemos una batería de 5000 mAh, en la gráfica a continuación se puede visualizar el tiempo de vida estimado en función de la permanencia de vehículos en una plaza del estacionamiento.

\insertimage[\label{calculo_bateria}]{secciones/section_4/images/calculo_bateria}{width=17cm}{Calculo teórico tiempo de vida de la batería}

En la sección de anexos, puede encontrarse el código para el cálculo y generación de la gráfica anterior. Ver anexo ~\ref{codigo_estimacion_bateria}

% Insertar Nodo de Borde
\clearpage

\subsubsection{Nodo de borde (concentrador)}
El objetivo de este nodo es poder recolectar los datos enviados a través de LoRa por los diferentes sensores, procesar los mismos y enviar la información al cloud para que esté disponible en la aplicación.

Para la elección como unidad de procesamiento, tuvimos en cuenta que los datos a ser procesados deben ser en tiempo real y con el mínimo tiempo de delay. Por lo cual el dispositivo de borde debía tener la capacidad de poder manejar esta carga. Dado que el equipo ya contaba con un mini ordenador utilizado ampliamente en la comunidad para prototipado (Raspberry Pi), hicimos pruebas de campo con resultados aceptables y decidimos utilizar el mismo para nuestro proyecto.

\insertimage[\label{raspberry_pi}]{secciones/section_4/images/raspberry_pi}{width=18cm}{Raspberry Pi 3 Model B+}

Como se puede apreciar en la figura anterior, Raspberry Pi es un mini ordenador, cuyo diseño de hardware es libre y cuenta con el sistema operativo GNU/Linux como Raspbian (aunque pueden utilizarse otros sistemas operativos basados en Linux como el caso de Ubuntu Server como ejemplo). 


Raspberry Pi utiliza una arquitectura de procesador ARM distinta a la que estamos acostumbrados a utilizar en nuestros ordenadores. Esta arquitectura es de tipo RISC (Reduced Instruction Set Computer), es decir, utiliza un sistema set de instrucciones realmente simple lo que le permite ejecutar tareas con un mínimo consumo de energía.

Además de lo mencionado anteriormente, este mini ordenador cuenta con salidas GPIO (General Purpose Input/Output - Entrada/Salida de Propósito General) lo que nos permite a través de la implementación del protocolo SPI poder dotar a la placa de comunicación LoRa. Y como esta cuenta con un chipset de WiFi integrado, resuelve el manejo y conexión a internet lo que nos habilita la integración con el backend de una forma sencilla.


\subsubsubsection{Sistema Operativo}
Raspberry Pi OS, previamente conocido como Raspbian, es el sistema operativo oficial diseñado específicamente para el microordenador Raspberry Pi. Esta distribución de Linux se basa en Debian y proporciona todo lo necesario para aprovechar al máximo esta placa. Está optimizado para funcionar en equipos ARM y viene con una amplia variedad de paquetes y programas preinstalados. Raspberry Pi OS utiliza el entorno de escritorio PIXEL (Pi Improved X-Window Environment, Lightweight), basado en LXDE, que es ligero y fácil de usar.

Existen tres ediciones diferentes de esta distribución:

\begin{itemize}
    \item Edición Completa: Incluye el escritorio PIXEL y una amplia selección de programas recomendados para comenzar a usar la distribución desde el primer momento.
    \item Edición Estándar: Incluye el escritorio y los programas básicos, sin ningún software recomendado adicional.
    \item Edición Lite: Es una imagen mínima basada en Debian que ocupa tan solo 400 MB. Esta edición incluye solo lo necesario para arrancar el dispositivo, dejando al usuario la responsabilidad de instalar los programas que necesite.
\end{itemize}

\subsubsubsection{PinOut}
A continuación se visualiza el pinout de la placa Raspberry Pi 3 Model B+:

\insertimage[\label{raspberry_pi_pinout}]{secciones/section_4/images/raspberry_pi_pinout}{width=18cm}{Pin Out - Raspberry Pi 3 Model B+}

Dado que nuestra Raspberry Pi 3 debe recibir y enviar mensajes a través de LoRa, se presenta a continuación el diagrama de conexión entre la Raspberry Pi y la placa Heltec. Para lograr esta comunicación, se utilizará el protocolo SPI (Interfaz Periférica en Serie) entre ambas placas:

\insertimage[\label{concentrator_schematic}]{secciones/section_4/images/concentrator_schematic}{width=17cm}{Conexión entre la Raspberry Pi y el módulo LoRa}

\subsubsubsection{Software}
Hemos creado una aplicación web de fácil uso que posibilita la configuración inicial del dispositivo. Esta configuración abarca aspectos como la conexión a la red WiFi, credenciales y parámetros de transmisión y recepción (Tx, Rx) LoRa.

Luego de la configuración inicial, el sistema queda listo para comenzar a recibir y procesar los datos obtenidos por los sensores. El código completo se encuentra disponible en el repositorio de github del proyecto, y a continuación ponemos la estructura principal que permite mostrar el funcionamiento básico del sistema.

A grandes rasgos podemos visualizar el diagrama de flujo de funcionamiento del software que corre la Raspberry Pi.

\insertimage[\label{nodo_colector_diagrama_flujos}]{secciones/section_4/images/nodo_colector_diagrama_flujos}{width=16cm}{Diagrama de flujos código concentrador}

% Insertar seccion sofware
\input{secciones/section_4/software_integral}
\clearpage
\section{Resultados}{\label{title:resultados}}
En este proyecto, abordamos la propuesta, diseño y desarrollo de un prototipo funcional de \textit{Smart Parking}, con el objetivo de brindar a los usuarios finales la capacidad de encontrar una plaza de estacionamiento cercana a su destino deseado mediante un sistema de reservas utilizando la aplicación que hemos construido.

En el capítulo anterior (Capítulo 4), se detalló el desarrollo de los cuatro módulos fundamentales del sistema:

\begin{itemize}
    \item Nodo sensor
    \item Nodo concentrador
    \item Sistema de Reservas (servidor)
    \item Aplicación Móvil
\end{itemize}

En este capítulo, nos enfocaremos en la validación y resultados obtenidos a través de las diferentes pruebas realizadas. Estas pruebas se han dividido en tres categorías:
\begin{itemize}
    \item Pruebas de Sensores
    \item Pruebas de Comunicación y Carga del Concentrador
    \item Integración de la Aplicación con el Sistema
\end{itemize}

A continuación, describiremos en detalle cada una de estas categorías de pruebas y presentaremos los hallazgos y resultados que surgieron durante el desarrollo del proyecto.

\subsection{Pruebas de Sensores}\label{title:pruebas_sensores}

Como se detalló en la sección de Elección de Sensores (ver: \ref{title:eleccion_sensores}), al validar los tipos de sensores a utilizar, llevamos a cabo un experimento en el garaje de uno de los miembros del equipo. Esto se realizó para medir el comportamiento de los sensores, y repetimos estas pruebas en varias ocasiones para obtener las muestras que se presentarán a continuación.

Para comenzar, montamos los sensores en nuestra placa de desarrollo. Iniciamos las pruebas utilizando vehículos estacionados de frente. Es importante señalar que, para esta sección, solo se mostrará cómo interactúa el sistema cuando el automóvil se estaciona de frente. El mismo experimento se repitió también para cuando el automóvil estaciona en reversa.

\begin{images}[\label{experimento_auto}]{Diferentes momentos del vehículo durante el experimento}
    \addimage[\label{experimento_auto:inicial}]{secciones/section_5/images/auto_momento_inicial}{width=8cm}{Posición inicial del vehículo}
    \addimage[\label{experimento_auto:ocupacion}]{secciones/section_5/images/auto_momento_ocupado_perspectiva}{width=8cm}{Vehículo estacionado}
    \addimage[\label{experimento_auto:libre}]{secciones/section_5/images/auto_momento_final}{width=8cm}{Vehículo en proceso de retirada}
\end{images}

En la fase inicial del experimento, el vehículo se encontraba en la posición que se muestra en la figura \ref{experimento_auto:inicial}. En este punto, el sensor de tiempo de vuelo (ToF) tenía un valor claro que indicaba que el objeto estaba fuera de rango. Este valor se representa en el recuadro A de la figura \ref{plot_labeled:vlx}. De manera similar, el magnetómetro mantenía un nivel constante del campo magnético, tal como se observa en \ref{plot_labeled:qmc}.

\begin{images}[\label{plot_labeled}]{Diferentes momentos del vehículo durante el experimento}
    \addimage[\label{plot_labeled:vlx}]{secciones/section_5/images/vl53X0_indicaciones}{width=8cm}{Muestras del sensor VL53L0X}
    \addimage[\label{plot_labeled:qmc}]{secciones/section_5/images/qmc_edited}{width=8cm}{Muestras del sensor QMC5883L}
\end{images}

A medida que el automóvil se posicionaba sobre el sensor (\ref{experimento_auto:ocupacion}), observamos que el sensor láser (\ref{plot_labeled:vlx}) comenzó a medir la distancia desde el sensor al chasis del vehículo. Esta distancia osciló alrededor de los $180 mm$, un valor constante que depende del tipo de vehículo debido a las diferentes alturas de los chasis. También observamos una fuerte variación en el campo magnético registrado por el magnetómetro, que luego se estabilizó, como se muestra en la situación descrita por el sensor láser. Ambas situaciones mencionadas aquí se refieren al recuadro \quotes{B} de ambas imágenes.

Finalmente, cuando el vehículo comenzó a retirarse (\ref{experimento_auto:libre}), nuevamente se detectaron variaciones, que se describen en el recuadro \quotes{C}.

Cabe destacar que las lecturas del magnetómetro, como se observa en la figura \ref{plot_labeled:qmc}, eran ruidosas, lo que dificultaba la definición de un umbral de detección. Fue durante este experimento que implementamos el \textit{Detector de Anomalías}, que se puede ver en \ref{detector_anomalias}.


\subsection{Pruebas de Comunicación y Carga del Concentrador}\label{title:cx_load_concentrador}



\subsection{Búsqueda y reserva de una plaza}
Para hacer una reserva el usuario puede navegar en el mapa, o bien utilizar la barra de búsqueda para introducir su destino. Una vez que el mismo esté seleccionado, se realiza una consulta en el backend para poder verificar cuales estacionamientos cuentan aún con disponibilidad y mostrarselos al usuario. Como puede verse en la Figura ~\ref{busqueda_process}.

\begin{images}[\label{busqueda_process}]{Proceso de busqueda de estacionamientos disponibles}
    \addimage[\label{busqueda_1}]{secciones/section_5/images/busqueda_1}{width=7cm}{}
    \addimage[\label{busqueda_2}]{secciones/section_5/images/busqueda_2}{width=7cm}{}
    \addimage[\label{busqueda_3}]{secciones/section_5/images/busqueda_3}{width=7cm}{}
\end{images}

Ya el usuario habiendo seleccionado el estacionamiento que desea, completa con el tipo de reserva que quiere elegir, teniendo la posibilidad de que la reserva sea fija según un esquema de tiempo definido o libre.

\begin{images}[\label{reserva_process}]{Selección del tipo de reserva (libre o limitada)}
    \addimage[\label{reserva_limitada}]{secciones/section_5/images/reserva_limitada}{width=4cm}{}
\end{images}

Finalmente se envía esa información al backend el cual valida e informa el estado de la reserva al usuario.

\begin{images}[\label{reservacion_checkout}]{Información sobre el estado de la reserva}
    \addimage[\label{reserva_checkout_1}]{secciones/section_5/images/reserva_checkout_1}{width=4cm}{}
    \addimage[\label{reserva_checkout_2}]{secciones/section_5/images/reserva_checkout_2}{width=4cm}{}
    \addimage[\label{reserva_checkout_4}]{secciones/section_5/images/reserva_checkout_4}{width=4cm}{}
\end{images}


% SUB-SECCIÓN
\subsection{Reservaciones en curso e historial}
En cualquier aplicación orientada a la reserva de servicios, resulta fundamental contar con una sección que permita a los usuarios visualizar tanto sus reservaciones actuales como las históricas. Esta función proporciona una perspectiva completa de las reservas realizadas y en curso, ofreciendo una experiencia integral de seguimiento y gestión de las mismas.

Basándonos en el contexto anteriormente descrito, hemos desarrollado un componente destinado a la visualización de estas reservas, el cual se puede apreciar en la imagen \ref{vista_reservaciones}

\begin{images}[\label{vista_reservaciones}]{Vista reservaciones en curso e historicas}
    \addimage[\label{reservacion_en_curso}]{secciones/section_5/images/reservacion_en_curso}{width=5cm}{}
    \addimage[\label{reservacion_historica}]{secciones/section_5/images/reservacion_historica}{width=5cm}{}
\end{images}

\subsubsection{Vista detallada de la reserva}
Tanto para las reservaciones en curso como para las historicas, al ingresar en un item el usuario puede tener una vista detallada de la reserva con algunos datos de interes y un codigo QR generado que podría ser utilizado a futuro en una automatización del ingreso como se detalla en \ref{automatizacion_ingreso}

\insertimage[\label{vista_detallada_reserva}]{secciones/section_5/images/vista_detallada_reserva}{width=5cm}{Vista detallada de la reserva}

\subsubsection{Cancelación de la reserva}

Finalmente, en caso de que el usuario cambie de opinión después de haber realizado la reserva, se le brinda la opción de cancelarla directamente desde la interfaz. En un entorno de producción, podría ser contemplado dentro del modelo de negocio implementar reembolsos en base al momento en que se efectúa la cancelación. Aunque este aspecto no se aborda en detalle en este proyecto final, se menciona como una nota relevante para los lectores interesados.

\begin{images}[\label{cancelacion_reserva}]{Proceso de cancelación de una reserva}
    \addimage[\label{cancelacion_1}]{secciones/section_5/images/cancelacion_1}{width=5cm}{}
    \addimage[\label{cancelacion_2}]{secciones/section_5/images/cancelacion_2}{width=5cm}{}
    \addimage[\label{cancelacion_3}]{secciones/section_5/images/cancelacion_3}{width=5cm}{}
\end{images}

\clearpage
\section{Potenciales Mejoras}
A continuación se adjuntan algunas potenciales mejoras que el proyecto podría tener en estadíos futuros.

% SUB-SECCIÓN
\subsection{Extensión del sistema a lugares públicos de estacionamiento}

Si bien el enfoque principal de nuestro proyecto fue utilizar el producto en estacionamientos privados, el hecho de poder encontrar también estacionamiento en zona pública contribuye a la disminución de tráfico en horas pico, ayuda a reducir emisiones de CO2, etc. 

La mejora que estamos proponiendo en este apartado cambia el modelo de negocio inicial. En este caso hablamos de extenderlos para que la información sea de uso gratuito en los lugares públicos de estacionamiento. A continuación listamos las potenciales mejoras que creemos deberían llevarse a cabo para lograr extender el mismo a lugares públicos, aunque no las desarrollaremos dado que no es el objetivo de este informe.

% SUB-SECCIÓN
\subsection{LoRa Gateway Industrial}
Para poder manejar eficientemente la escalabilidad del sistema en usos públicos, se debería incorporar un Gateway LoRa industrial como el que se visualiza en la figura XX 

% CAMBIAR POR FOTOS DEL LORA GATEWAY
\insertimage[\label{lora_gateway}]{secciones/section_6/images/lora_gateway}{width=7.3cm}{Gateway Industrial Lora}

Este dispositivo es un elemento central en una red LoraWAN. La función de un Gateway es generar la red inalámbrica LoRaWAN para cobertura a los nodos, comunicar con los nodos para recibir la información que estos mandan o para transmitirles los comandos pertinentes y, por último, comunicarse con el Network Server aguas arriba (Para este último paso ya se utiliza una comunicación TCP/IP). 

Esto inicialmente no estaba en el proyecto porque para el caso de redes privadas pequeñas como era el caso de uso nuestro, el gateway puede omitirse dado que nuestra placa raspberry pi utiliza LoRa directamente en la comunicación. Pero para el caso de redes extensas, deben utilizarse varios Gateways para asegurar la redundancia y la entrega exitosa de los datos. A continuación se presenta una ilustración de cómo influye el gateway en una red de sensores: 

% CAMBIAR POR FOTOS LORA WAN
\insertimage[\label{lorawan}]{secciones/section_6/images/lorawan}{width=14cm}{Arquitectura LoraWAN}

% SUB-SECCIÓN
\subsection{Integración con sistema de pagos electrónicos}

Pensando este sistema en un entorno de producción, el mismo debe contar con un sistema integrados de pagos electrónicos como podría ser Mercado Pago para eliminar la necesidad de efectivo o tarjetas físicas. Además, en términos de eficiencia operativa esto, simplifica la recolección de pagos para los operadores de estacionamientos. Los pagos se registrarían automáticamente en la plataforma, lo que reduciría la carga administrativa y minimizaría los errores manuales.

% SUB-SECCIÓN
\subsection{Automatización sistema de ingreso}{\label{automatizacion_ingreso}}
La automatización del acceso a través de la lectura de patentes vehiculares y códigos QR generados en la aplicación, busca simplificar el proceso de ingreso y salida de vehículos en estacionamientos, eliminando la necesidad de un operador humano y optimizando la eficiencia en la gestión del estacionamiento. La implementación de esta mejora conlleva múltiples beneficios. En primer lugar, mejora la experiencia del usuario al minimizar el tiempo de espera entre la entrada y la salida del estacionamiento. Además, al eliminar la necesidad de personal humano en las barreras, se reducen los costos operativos y se aumenta la disponibilidad del sistema durante todo el día.

% SUB-SECCIÓN
\subsection{Creación de zona WiFi para la configuración del sensor}{\label{potencial_mejora_wifiOTA}}
Como potencial futura mejora también, añadimos la viabilidad de incorporar la funcionalidad de creación de una red WiFi temporal
diseñada específicamente para la configuración de nodos sensores. Esta iniciativa, conocida como WiFi OTA (Over-The-Air), 
abriría la posibilidad de llevar a cabo la configuración de sensores de manera inalámbrica y directa mediante un smartphone u otro dispositivo compatible.

Esta propuesta se fundamenta en las capacidades del chip ESP32, el cual dispone de un módulo WiFi integrado. La implementación requeriría
no solo la habilidad para establecer una red ad-hoc entre el nodo sensor y el dispositivo configurador, sino también la creación de un servidor HTTP
que deberá estar alojado en la ESP32, que permita la visualización y ajuste de los parámetros de configuración.Para obtener más información de referencia, se
pueden explorar las implementaciones ya existentes, como las proporcionadas por la librería WiFiManager.

Al permitir la configuración inalámbrica de sensores, se podría agilizar y simplificar el proceso de despliegue y ajuste de los dispositivos,
mejorando la experiencia del usuario y la eficiencia operativa en situaciones donde la configuración por cable puede resultar poco práctica o inviable.

\clearpage
\section{Conclusiones}
En este último capítulo, se presentan las conclusiones finales del trabajo realizado, además se exponen las limitaciones y problemas encontrados en el diseño y configuración de los equipos, junto con las interpretaciones de las pruebas realizadas.

En la concepción de la arquitectura propuesta, hemos explorado diversas perspectivas en cuanto a la integración de módulos diversos, priorizando principios de escalabilidad, seguridad y descentralización. En consecuencia, hemos concebido y desarrollado una arquitectura para sistemas de estacionamiento inteligente, aprovechando las capacidades de la computación en la nube. Hemos abordado metodologías innovadoras para la gestión dinámica de la información en la base de datos (DynamoDB), lo cual otorga la flexibilidad necesaria para la incorporación de nuevas funcionalidades y la adaptación de componentes futuros al sistema.

Hemos conseguido desarrollar un prototipo IoT que detecta la presencia de vehículos y manda esa información a través de LoRa hacia el nodo de borde. En este punto, también pudimos proveer de seguridad y autenticación con credenciales únicas para cada nodo asegurando la información que se envía al cloud. Junto con lo anterior hemos desarrollado una aplicación móvil para la interacción del usuario con el sistema, proporcionando una experiencia fluida y efectiva como si se tratase de un producto profesional.

La aplicación móvil ha sido testeada en una variedad de dispositivos Android y emuladores, para poder demostrar su robustez en términos de comunicación y desempeño. La arquitectura en su conjunto ha revelado su versatilidad y capacidad para gestionar de manera eficiente las demandas de tráfico entrante en entornos de testeo.

Este proyecto destaca la eficacia de las plataformas tipo PaaS (Platform as a Service), las cuales ofrecen un entorno integral para la implementación de sistemas IoT. Estas plataformas solventan cuestiones técnicas y de seguridad, aseguran la conectividad de extremo a extremo, garantizan la calidad del servicio y posibilitan una escalabilidad sostenible a largo plazo. Además, abren la posibilidad de adoptar modelos de pago por uso en el futuro, lo que contribuiría a la optimización de los recursos y a la reducción de costos innecesarios.

No obstante, es crucial reconocer que la creación de un prototipo funcional ha conllevado desafíos sustanciales en términos de investigación, cálculos de batería e implementación. Para llevar esta iniciativa a la etapa de producto, es necesario continuar con estudios profundos, como la sustitución del microcontrolador principal por una alternativa de menor consumo energético y la optimización del diseño del PCB a través del empleo de chips de montaje superficial (SMD). Además, se requiere un análisis exhaustivo de los materiales para la selección adecuada de la carcasa del dispositivo, garantizando su resistencia ante posibles aplastamientos accidentales.

Por último, la ejecución de este proyecto nos permitió aplicar y reforzar conocimientos adquiridos a lo largo de la carrera, como son: programación en C, arquitectura de sistemas embebidos, conceptos de cloud computing vistos en materia como sistemas digitales avanzados, entre otros.

\clearpage

\begin{appendixd}
    \section{Repositorios de Código}
        \subsection{Código nodo sensor}{\label{codigo_sensor}}
            El código de nuestro nodo sensor se encuentra completamente desarrollado en C haciendo uso del framework del fabricante
            Expressif. Aprovechando lo aprendido en las materias de Sistemas Digitales hemos podido aplicar en la construcción de nuestro
            nodo sensor los conceptos relacionados al manejo y programación de microcontroladores, protocolos de comunicación como I2C, SPI, etc.
            manejo de los modos de energía del microcontrolador, etc. En el link adjunto se puede ver el código del nodo sensor.


            Link: \url{repositorio_codigo}

        \subsection{Estimación tiempo de vida de la bateria}{\label{codigo_estimacion_bateria}}
            Como se menciono en anteriormente en la seccion ~\ref{sec:bateria_del_nodo_sensor} dada la naturaleza de nuestro
            proyecto, es fundamental realizar los calculos correctos para que el consumo de la batería sea el minímo deseado 
            y poder así maximir el tiempo de vida de esta y del nodo. El código de dicha estimación se encuentra disponible
            en el siguiente repositorio de GitHub abierto publicamente.

            Link: \url{repositorio_codigo}


        \subsection{Código generación mapa de Rosario indexado}
            Este anexo contiene el código en lenguaje Python, utilizado para la generación del mapa de Rosario. 
            Parte del mismo es utilizado en el backend cuando se registra un estacionamiento nuevo para poder guardar su latitud y 
            longitud con un hash de H3 correspondiente y que esté disponible cuando un usuario busque un estacionamiento determinado.

            Link: \url{repositorio_codigo}

\end{appendixd}




\clearpage
\begin{references}
    \bibitem{ref1}
    Urban mobility at a tipping point. Disponible: \url{https://www.mckinsey.com/business-functions/sustainability/our-insights/urban-mobility-at-a-tipping-point}

    \bibitem{ref2}
    Parking and Sustainable Urban Mobility Planning. Disponible: 
    \url{https://www.eltis.org/sites/default/files/parking_and_sustainable_urban_mobility_planning.pdf}

    \bibitem{ref3}
    New Smart Parking System Based on Resource Allocation. Disponible: 
    \url{https://www.researchgate.net/publication/229033950_New_Smart_Parking_System_Based_on_Resource_Allocation_and_Reservations}

    \bibitem{ref4}
    A Reservation-based Smart Parking System. Disponible: 
    \url{https://digitalcommons.unl.edu/cgi/viewcontent.cgi?article=1045&context=computerscidiss}

    \bibitem{ref5}
    IoT-Based Parking Monitoring System. Disponible: 
    \url{https://iopscience.iop.org/article/10.1088/1755-1315/794/1/012134/pdf}

    \bibitem{ref6}
    Evaluating LoRaWAN for IoT applications. Disponible: 
    \url{http://www.diva-portal.org/smash/get/diva2:1245767/FULLTEXT01.pdf}

    \bibitem{ref7}
    Smart Parking Sensors: State of the Art and Performance Evaluation. Disponible: 
    \url{https://www.researchgate.net/publication/340250091_Smart_Parking_Sensors_State_of_the_Art_and_Performance_Evaluation}

    \bibitem{ref8}
    Survey of Smart Parking Systems. Disponible: 
    \url{https://ri.conicet.gov.ar/bitstream/handle/11336/153854/CONICET_Digital_Nro.605cad6a-fcaa-4ace-b269-957e54ea4629_A.pdf?sequence=2&isAllowed=y}

    \bibitem{ref9}
    An Internet of Things (IOT) based Smart Parking Routing System for Smart Cities. Disponible: 
    \url{https://www.academia.edu/41524293/An_Internet_of_Things_IOT_based_Smart_Parking_Routing_System_for_Smart_Cities}

    \bibitem{ref10}
    El estado de la red. Banco Interamericano de Desarrollo (BID). Disponible: 
    \url{https://conexionintal.iadb.org/2017/03/08/el-estado-de-la-red/}
    
    \bibitem{ref11}
    Wi-Fi is an essential IoT enabler. Disponible:
    \url{https://www.wi-fi.org/discover-wi-fi/internet-things}

    \bibitem{ref13}
    ¿Qué es el Internet de las cosas (IoT)?. Disponible:
    \url{https://www.redhat.com/es/topics/internet-of-things/what-is-iot}

    \bibitem{ref14}
    ¿Qué es Wi-Fi?. Disponible:
    \url{https://www.cisco.com/c/es_mx/products/wireless/what-is-wifi.html}

    \bibitem{ref15}
    What Is IoT Connectivity: A Comparison Guide. Disponible:
    \url{https://www.emnify.com/blog/iot-connectivity}

    \bibitem{ref51}
    “SigFox”. \textit{Low power wide area network (LPWAN) dedicated to Massive IoT} Disponible: \url{https://www.sigfox.com/}

    \bibitem{ref52}
    “SigFox”. \textit{Que es?; Introducción; Alcance y Desarrollo} Disponible: \url{https://sigfox.com.py/que-es-sigfox/}

    \bibitem{ref53}
    “BLE”. \textit{Bluetooth Low Energy} Disponible: \url{https://en.wikipedia.org/wiki/Bluetooth_Low_Energy}

    \bibitem{ref16}
    Comparision of LoRa and NB-IoT in Terms of Connectivity\\
    %\url{https://www.diva-portal.org/smash/get/diva2:1453316/FULLTEXT01.pdf}

    \bibitem{ref17}
    Implementación de un GateWay para conectar sensores con transceiver de LoRa a servicios públicos y privados\\
    %\url{https://www.redhat.com/es/topics/internet-of-things/what-is-iot}

    \bibitem{ref18}
    Study of LoRaWAN device and gateway setups. Disponible:
    \url{https://www.diva-portal.org/smash/get/diva2:1679773/FULLTEXT01.pdf}

    \bibitem{ref19}
    M. O. Farooq y D. Pesch, \textit{“Analyzing lora: A use case perspective,” in 2018 IEEE 4th World Forum on
    Internet of Things (WF-IoT). IEEE, 2018, pp. 355\-360.}\\
    %\url{https://www.redhat.com/es/topics/internet-of-things/what-is-iot}

    \bibitem{ref20}
    S. Daud, T. S. Yang, M. A. Romli, Z. A. Ahmad, N. Mahrom, y R. A. A. Raof,\textit{“Performance Evaluation of Low Cost LoRa Modules in IoT Applications,” IOP Conference Series:
    Materials Science and Engineering, vol. 318, p. 012053, Mar. 2018}. Disponible: 
    \url{https://iopscience.iop.org/article/10.1088/1757-899X/318/1/012053}    

    \bibitem{ref21}
    H. Boyes, B. Hallaq, J. Cunningham, y T. Watson, \textit{“The industrial internet of things (IIoT): An
    analysis framework,” Computers in Industry, vol. 101}. Disponible: 
    \url{https://linkinghub.elsevier.com/retrieve/pii/S0166361517307285}  
    

    \bibitem{ref22}
    “WIFI LoRa 32 (V2).” \url{https://heltec.org/project/wifi-lora-32}

    \bibitem{ref45}
    “ESP32 Datasheet.” \url{https://www.espressif.com/sites/default/files/documentation/esp32_datasheet_en.pdf}

    \bibitem{ref46}
    “Espressif.” \url{https://www.espressif.com/en/products/socs/esp32}

    \bibitem{ref44}
    "ESP32 the Internet of Things with ESP32". Disponible: \url{http://esp32.net/#Development}

    \bibitem{ref23}
    “F. R. Pi, “Raspberry Pi 3 Model B+” \url{https://www.raspberrypi.org/products/raspberry-pi-3-model-b}
    
    \bibitem{ref24}
    LoRa-Alliance, “About LoRaWAN® | LoRa Alliance®,” 2020. \url{https://lora-alliance.org/about-lorawan}

    \bibitem{ref25}
    V. Cola, “ESP32SingleChannelGateway,”. Disponible: \url{https://github.com/vpcola/ESP32SingleChannelGateway}

    \bibitem{ref26}
    “SX1278 | 137MHz to 525MHz Long Range Low Power Transceiver | Semtech.” Disponible: \url{https://www.semtech.com/products/wireless-rf/lora-transceivers/sx1278}

    \bibitem{ref27}
    C. L. Carrión, “Evaluación del rango de transmisión de LoRa para redes de sensores inalámbricos con LoRaWAN en ambientes urbanos,” Tesis de grado, Universidad de Cuenca, Cuenca, Ecuador, 2018

    \bibitem{ref28}
    L. Tessaro, C. Raffaldi, M. Rossi, y D. Brunelli, “LoRa Performance in Short Range Industrial Applications,” in 2018 International Symposium on Power Electronics, Electrical Drives, Automation and Motion (SPEEDAM)

    \bibitem{ref29}
    “ZigBee”. Disponible: \url{https://zigbee.org/}

    \bibitem{ref31}
    “LoraDesignGuide\_std.pdf.” Disponible: \url{https://www.semtech.com/products/wireless-rf/lora-connect/sx1276}

    \bibitem{ref32}
    “HMC5883L Triple Axis Magnetometer Breakout”. Disponible: \url{http://www.geeetech.com/wiki/index.php/HMC5883L_Triple_Axis_Magnetometer_Breakout}

    \bibitem{ref33}
    “HMC5883L Magnetometer datasheet”. Disponible: \url{https://www.farnell.com/datasheets/1683374.pdf}

    \bibitem{ref34}
    “HC-SR04 Ultrasonic Sensor Module User Guide”. Disponible: \url{https://www.handsontec.com/dataspecs/HC-SR04-Ultrasonic.pdf}

    \bibitem{ref35}
    “VL53L0X Datasheet”. Disponible: \url{https://pdf1.alldatasheet.com/datasheet-pdf/view/948120/STMICROELECTRONICS/VL53L0X.html}

    \bibitem{ref36}
    “Flutter Dev/Docs”. \textit{Internationalizing Flutter apps, Adding assets and images, state management, etc} Disponible: \url{https://flutter.dev/}

    \bibitem{ref37}
    “Cloud computing: Issues and challenges”. Disponible: \url{https://doi.org/10.1109/AINA.2010.187}

    \bibitem{ref38}
    “J. Peng, X. Zhang, Z. Lei, B. Zhang, W. Zhang, and Q. Li, Comparison of several cloud computing platforms”. Disponible: \url{https://doi.org/10.1109/ISISE.2009.94}

    \bibitem{ref39}
    “Cloud computing: A perspective study”. Disponible: \url{http://dblp.uni-trier.de/db/journals/ngc/ngc28.html#WangLYHKTF10}

    \bibitem{ref40}
    “Cloud Computing: Comparison and Analysis of Cloud Service Providers-AWs, Microsoft and Google”. Disponible: \url{https://ieeexplore.ieee.org/document/9337100}

    \bibitem{ref41}
    “Cloud computing service providers: A comparative study”. Disponible: \url{https://www.iasj.net/iasj/download/280f00e221c1a458}

    \bibitem{ref42}
    “Cloud Computing Architecture”. Disponible: \url{https://binalkagathara.medium.com/comparative-study-of-cloud-computing-and-mobile-cloud-computing-1878ae6a2d94}

    \bibitem{ref43}
    “Cloud Computing: Architecture, Services, Deployment Models, Storage, Benefits and Challenges”. Disponible: \url{https://www.researchgate.net/publication/341788106_Cloud_Computing_Architecture_Services_Deployment_Models_Storage_Benefits_and_Challenges}

    \bibitem{ref48}
    “H3: Uber\'s Hexagonal Hierarchical Spatial Index”. Disponible: \url{https://www.uber.com/blog/h3/}

    \bibitem{ref49}
    “H3: A Hexagonal Hierarchical Geospatial Indexing System”. Disponible: \url{https://github.com/uber/h3}

    \bibitem{ref50}
    “S2 Geometry”. \textit{Library for spherical geometry that aims to have the same robustness, flexibility, and performance as the very best planar geometry libraries}. Disponible: \url{https://s2geometry.io/}

    \bibitem{ref55}
    “AWS”. \textit{Amazon Web Services}. Disponible: \url{https://docs.aws.amazon.com/}

    \bibitem{ref56}
    “DrawIO”. \textit{Diagrama software}. Disponible: \url{https://app.diagrams.net/}

    \bibitem{ref57}
    “Which types of batteries for your IoT devices?”. Disponible: \url{https://www.saft.com/energizing-iot/types-batteries-iot-devices}

    \bibitem{ref58}
    “Battery-less IoT Devices”. Disponible: \url{http://www.diva-portal.org/smash/get/diva2:1651573/FULLTEXT01.pdf}

    \bibitem{ref59}
    “Nikesh Man Shakya. Design and development of energy-efficient transmission for wireless IoT modules. Networking and Internet Architecture [cs.NI]. Université Paris-Saclay, 2019. English. ffNNT :
    2019SACLL001ff. fftel-02071262v2f”.

    \bibitem{ref60}
    “Why Flutter is the most popular cross-platform mobile SDK”. Disponible: \url{https://stackoverflow.blog/2022/02/21/why-flutter-is-the-most-popular-cross-platform-mobile-sdk/}

    \bibitem{ref61}
    “Tech Stacks to Launch Your App Development into Stratosphere”. Disponible: \url{https://appinventiv.com/blog/technology-for-mobile-app-development/}

    \bibitem{ref62}
    “Arquitectura Flutter”. Disponible: \url{https://devjaime.medium.com/d%C3%ADa-3-arquitectura-flutter-y-configuraci%C3%B3n-del-proyecto-9307aa98b4fe}

\end{references}


% FIN DEL DOCUMENTO
\end{document}