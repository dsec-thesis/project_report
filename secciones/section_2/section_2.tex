\newpage
\section{Antecedentes y trabajo relacionado}
En este trabajo final de grado, nos enfocaremos principalmente en el desarrollo de un sistema de estacionamiento inteligente que permita asistir (principalmente) a los conductores para encontrar lugares libres en un distrito de estacionamientos específicos. En conjunto con esto, un objetivo importante del sistema es ayudar a reducir el tiempo de búsqueda de un lugar, lo que se traduce en una disminución de contaminación de CO2 y también de congestión del tráfico.

En este capítulo, revisaremos el background de los \textit{Smart Parking Systems} actuales y los desafíos como así también discutiremos brevemente el trabajo relacionado.

% SUB-SECCIÓN
\subsection{Estado del arte, crítica y propuesta}
El problema de enrutamiento de vehículos (VRP) es uno de los problemas más estudiados en los sistemas de transportes inteligentes (ITS) en la actualidad. El crecimiento de las ciudades en términos de población y número de vehículos ha llevado a la búsqueda de alternativas para mejorar el sistema de transporte. La gran cantidad de vehículos intentando acceder a áreas que permanecen permanentemente congestionadas significa que encontrar un espacio para estacionar a menudo es difícil, lo que genera aún más caos en el tráfico, mayor consumo de combustible y, por lo tanto, mayores emisiones de gases de efecto invernadero.

Para aliviar dicha congestión de tráfico y mejorar la comodidad de los conductores, muchos sistemas de estacionamiento inteligentes tienen como objetivo satisfacer a las partes involucradas. Los sistemas actuales de estacionamiento inteligente o sistemas de orientación de estacionamiento tienen la siguiente clasificación: 

\begin{itemize}
    \item Búsqueda a ciegas: es una estrategia básica que los conductores emplean cuando carecen de información sobre estacionamientos disponibles. En esta modalidad, los conductores siguen buscando espacios de estacionamiento dentro de una cierta distancia a su destino. Si encuentran un espacio disponible, dejan de buscar; de lo contrario, amplían el área de búsqueda y continúan buscando espacios vacantes en los estacionamientos cercanos de forma constante.
    \item Parking Information Sharing (PIS): Este mecanismo se adopta comúnmente en el diseño actual de los sistemas inteligentes de estacionamiento \cite{ref7}. Después de que el sistema de estacionamiento inteligente publique la información de disponibilidad de estacionamiento a los conductores en cierta área, el conductor decidirá su destino de estacionamiento deseado donde el estacionamiento tenga espacios disponibles, según la información de disponibilidad de estacionamiento obtenida. Sin embargo, si el número de espacios vacantes en un estacionamiento es muy limitado en horas ocupadas, es probable que aumente el número de conductores que demandan estos espacios de estacionamiento según la información de estacionamiento. Este fenómeno se conoce como "múltiples autos persiguiendo un solo espacio", lo que puede causar una congestión grave.
    \item Buffered PIS (BPIS): Para abordar el problema del fenómeno "múltiples autos persiguiendo un solo espacio", algunos diseñadores de sistemas inteligentes de estacionamiento modifican el mecanismo PIS. Reducen intencionalmente el número de espacios vacantes al publicar la información de disponibilidad en tiempo real, para mantener un buffer. Por lo tanto, aunque puede haber más conductores persiguiendo los espacios limitados disponibles, el sistema tiene algunos espacios adicionales para evitar conflictos. Sin embargo, es difícil determinar el número de espacios en el buffer. Si el buffer es demasiado pequeño, el problema de "múltiples autos persiguiendo un solo espacio" no se eliminará. Si es demasiado grande, la utilización de los espacios de estacionamiento será baja.
\end{itemize}

Como se mencionó anteriormente, el sistema de búsqueda a ciegas es un sistema de bucle abierto en el que los usuarios toman decisiones sin considerar el estado del sistema. Las estrategias PIS y BPIS permiten a los conductores tomar decisiones basadas en el estado del sistema (por ejemplo, información de disponibilidad de estacionamiento). Sin embargo, los fenómenos de múltiples autos persiguiendo un solo espacio no se pueden eliminar por completo y muchas veces guían a los conductores a sus destinos de estacionamiento “deseados” empeorando la situación, dado que no son lo suficientemente “inteligentes”. Por lo tanto, es muy deseable proporcionar una estrategia efectiva para abordar estas preocupaciones.

En este proyecto, diseñamos e implementamos el prototipo de un Sistema de Estacionamiento Inteligente basado en Reservas  que no solo transmite información de estacionamiento en tiempo real a los conductores como parte de una aplicación, sino que también ofrece un servicio de reserva como parte de un servicio dirigido al usuario. Basado en técnicas avanzadas de sensores y comunicación móvil, este sistema procesa flujos de datos de sensores con marca de tiempo de la red del estacionamiento y publica esta información de disponibilidad en la aplicación a diseñar. Luego los conductores pueden obtener esta información desde sus smartphones, por lo tanto si un conductor hace una reserva con éxito, garantiza un espacio de estacionamiento disponible para él. Y el conductor puede estacionarse en el espacio reservado sin necesidad de buscar.