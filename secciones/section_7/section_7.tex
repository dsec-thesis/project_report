\clearpage
\section{Conclusiones}
En este último capítulo, se presentan las conclusiones finales del trabajo realizado, además se exponen las limitaciones y problemas encontrados en el diseño y configuración de los equipos, junto con las interpretaciones de las pruebas realizadas.

En la concepción de la arquitectura propuesta, hemos explorado diversas perspectivas en cuanto a la integración de módulos diversos, priorizando principios de escalabilidad, seguridad y descentralización. En consecuencia, hemos concebido y desarrollado una arquitectura para sistemas de estacionamiento inteligente, aprovechando las capacidades de la computación en la nube. Hemos abordado metodologías innovadoras para la gestión dinámica de la información en la base de datos (DynamoDB), lo cual otorga la flexibilidad necesaria para la incorporación de nuevas funcionalidades y la adaptación de componentes futuros al sistema.

Hemos conseguido desarrollar un prototipo IoT que detecta la presencia de vehículos y manda esa información a través de LoRa hacia el nodo de borde. En este punto, también pudimos proveer de seguridad y autenticación con credenciales únicas para cada nodo asegurando la información que se envía al cloud. Junto con lo anterior hemos desarrollado una aplicación móvil para la interacción del usuario con el sistema, proporcionando una experiencia fluida y efectiva como si se tratase de un producto profesional.

La aplicación móvil ha sido testeada en una variedad de dispositivos Android y emuladores, para poder demostrar su robustez en términos de comunicación y desempeño. La arquitectura en su conjunto ha revelado su versatilidad y capacidad para gestionar de manera eficiente las demandas de tráfico entrante en entornos de testeo.

Este proyecto destaca la eficacia de las plataformas tipo PaaS (Platform as a Service), las cuales ofrecen un entorno integral para la implementación de sistemas IoT. Estas plataformas solventan cuestiones técnicas y de seguridad, aseguran la conectividad de extremo a extremo, garantizan la calidad del servicio y posibilitan una escalabilidad sostenible a largo plazo. Además, abren la posibilidad de adoptar modelos de pago por uso en el futuro, lo que contribuiría a la optimización de los recursos y a la reducción de costos innecesarios.

No obstante, es crucial reconocer que la creación de un prototipo funcional ha conllevado desafíos sustanciales en términos de investigación, cálculos de batería e implementación. Para llevar esta iniciativa a la etapa de producto, es necesario continuar con estudios profundos, como la sustitución del microcontrolador principal por una alternativa de menor consumo energético y la optimización del diseño del PCB a través del empleo de chips de montaje superficial (SMD). Además, se requiere un análisis exhaustivo de los materiales para la selección adecuada de la carcasa del dispositivo, garantizando su resistencia ante posibles aplastamientos accidentales.

Por último, la ejecución de este proyecto nos permitió aplicar y reforzar conocimientos adquiridos a lo largo de la carrera, como son: programación en C, arquitectura de sistemas embebidos, conceptos de cloud computing vistos en materia como sistemas digitales avanzados, entre otros.
