\clearpage

\begin{appendixd}
    \section{Repositorios de Código}
        \subsection{Código nodo sensor}{\label{codigo_sensor}}
            El código de nuestro nodo sensor se encuentra completamente desarrollado en C haciendo uso del framework del fabricante
            Expressif. Aprovechando lo aprendido en las materias de Sistemas Digitales hemos podido aplicar en la construcción de nuestro
            nodo sensor los conceptos relacionados al manejo y programación de microcontroladores, protocolos de comunicación como I2C, SPI, etc.
            manejo de los modos de energía del microcontrolador, etc. En el link adjunto se puede ver el código del nodo sensor.


            Link: \url{https://github.com/dsec-thesis/sensor}

        \subsection{Código nodo concentrador}{\label{codigo_concentrador}}
            El módulo del nodo concentrador ha sido diseñado en Python, aprovechando su 
            flexibilidad y facilidad de programación para cumplir con sus funciones esenciales.
            Este componente desempeña un papel crucial en la arquitectura, siendo el puente entre 
            la información de los datos transmitidos por los nodos sensores a través de la tecnología
            LoRa hacia el backend.


            Link: \url{https://github.com/dsec-thesis/concentrator}

        \subsection{Estimación tiempo de vida teórico de la bateria}{\label{codigo_estimacion_bateria}}
            Como se menciono en anteriormente en la seccion ~\ref{sec:bateria_del_nodo_sensor} dada la naturaleza de nuestro
            proyecto, es fundamental realizar los calculos correctos para que el consumo de la batería sea el minímo deseado 
            y poder así maximir el tiempo de vida de esta y del nodo. El código de dicha estimación se encuentra disponible
            en el siguiente repositorio de GitHub abierto publicamente.

            Link: \url{https://github.com/dsec-thesis/project_report/tree/main/utils/battery_lifetime_estimation.ipynb}


        \subsection{Código generación mapa de Rosario indexado}
            Este anexo contiene el código en lenguaje Python, utilizado para la generación del mapa de Rosario. 
            Parte del mismo es utilizado en el backend cuando se registra un estacionamiento nuevo para poder guardar su latitud y 
            longitud con un hash de H3 correspondiente y que esté disponible cuando un usuario busque un estacionamiento determinado.

            Link: \url{https://github.com/dsec-thesis/project_report/tree/main/utils/rosario_h3_index.ipynb}

        \subsection{Código del backend}
            El link a continuación contiene el código del backend desplegado en AWS.

            Link: \url{https://github.com/dsec-thesis/backend}

        \subsection{Código aplicación móvil}
            El link a continuación conitene el código de la aplicación desarrollada en Flutter

            Link: \url{https://github.com/dsec-thesis/application}
    \clearpage
    \section{WiFi OTA (Over-The-Air)}{\label{configuracion_wifi}}
        WiFi OTA (Over-The-Air) se refiere a la capacidad de actualizar, configurar o gestionar dispositivos electrónicos de manera inalámbrica, 
        sin necesidad de conexiones físicas o cables. Esta tecnología ha cobrado gran relevancia en el mundo de la electrónica y 
        especialmente en el Internet de las Cosas (IoT), donde la conectividad y la capacidad de actualización remota son 
        esenciales para el funcionamiento y la escalabilidad de los dispositivos. Ademas, la importancia de WiFi OTA radica 
        en su capacidad para facilitar la administración y el mantenimiento de dispositivos distribuidos en ubicaciones diversas,
        sin la necesidad de acceso físico a cada uno de ellos. Esto presenta varias ventajas clave:

        \begin{itemize}
            \item Actualizaciones Remotas: La habilidad de actualizar el firmware y el software de los dispositivos de manera inalámbrica permite corregir errores, 
            agregar nuevas características y mejorar la seguridad sin requerir la presencia física del dispositivo. Esto es 
            fundamental en sistemas distribuidos y de gran escala.
            \item Ahorro de Tiempo y Recursos: La actualización y configuración inalámbrica ahorran tiempo y recursos al 
            eliminar la necesidad de acceder manualmente a cada dispositivo, lo que sería especialmente costoso en 
            despliegues masivos o en ubicaciones de difícil acceso.
            \item Actualización de Firmware: Los dispositivos IoT suelen requerir actualizaciones periódicas para corregir errores, mejorar la seguridad y 
            añadir nuevas funcionalidades. WiFi OTA permite realizar estas actualizaciones de manera eficiente y sin interrupciones.
            \item Configuración Inicial: En sistemas IoT, la configuración inicial de los dispositivos puede ser compleja. WiFi OTA posibilita la 
            configuración remota y simplificada de múltiples dispositivos al mismo tiempo.
            \item etc
        \end{itemize}



\end{appendixd}
