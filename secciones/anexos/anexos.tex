\clearpage

\begin{appendixd}
    \section{Repositorios de Código}
        \subsection{Código nodo sensor}{\label{codigo_sensor}}
            El código de nuestro nodo sensor se encuentra completamente desarrollado en C haciendo uso del framework del fabricante
            Expressif. Aprovechando lo aprendido en las materias de Sistemas Digitales hemos podido aplicar en la construcción de nuestro
            nodo sensor los conceptos relacionados al manejo y programación de microcontroladores, protocolos de comunicación como I2C, SPI, etc.
            manejo de los modos de energía del microcontrolador, etc. En el link adjunto se puede ver el código del nodo sensor.


            Link: \url{repositorio_codigo}

        \subsection{Estimación tiempo de vida de la bateria}{\label{codigo_estimacion_bateria}}
            Como se menciono en anteriormente en la seccion ~\ref{sec:bateria_del_nodo_sensor} dada la naturaleza de nuestro
            proyecto, es fundamental realizar los calculos correctos para que el consumo de la batería sea el minímo deseado 
            y poder así maximir el tiempo de vida de esta y del nodo. El código de dicha estimación se encuentra disponible
            en el siguiente repositorio de GitHub abierto publicamente.

            Link: \url{repositorio_codigo}


        \subsection{Código generación mapa de Rosario indexado}
            Este anexo contiene el código en lenguaje Python, utilizado para la generación del mapa de Rosario. 
            Parte del mismo es utilizado en el backend cuando se registra un estacionamiento nuevo para poder guardar su latitud y 
            longitud con un hash de H3 correspondiente y que esté disponible cuando un usuario busque un estacionamiento determinado.

            Link: \url{repositorio_codigo}

\end{appendixd}
