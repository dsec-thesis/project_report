\begin{references}
    \bibitem{ref1}
    Urban mobility at a tipping point. Disponible: \url{https://www.mckinsey.com/business-functions/sustainability/our-insights/urban-mobility-at-a-tipping-point}

    \bibitem{ref2}
    Parking and Sustainable Urban Mobility Planning. Disponible: 
    \url{https://www.eltis.org/sites/default/files/parking_and_sustainable_urban_mobility_planning.pdf}

    \bibitem{ref3}
    New Smart Parking System Based on Resource Allocation. Disponible: 
    \url{https://www.researchgate.net/publication/229033950_New_Smart_Parking_System_Based_on_Resource_Allocation_and_Reservations}

    \bibitem{ref4}
    A Reservation-based Smart Parking System. Disponible: 
    \url{https://digitalcommons.unl.edu/cgi/viewcontent.cgi?article=1045&context=computerscidiss}

    \bibitem{ref5}
    IoT-Based Parking Monitoring System. Disponible: 
    \url{https://iopscience.iop.org/article/10.1088/1755-1315/794/1/012134/pdf}

    \bibitem{ref6}
    Evaluating LoRaWAN for IoT applications. Disponible: 
    \url{http://www.diva-portal.org/smash/get/diva2:1245767/FULLTEXT01.pdf}

    \bibitem{ref7}
    Smart Parking Sensors: State of the Art and Performance Evaluation. Disponible: 
    \url{https://www.researchgate.net/publication/340250091_Smart_Parking_Sensors_State_of_the_Art_and_Performance_Evaluation}

    \bibitem{ref8}
    Survey of Smart Parking Systems. Disponible: 
    \url{https://ri.conicet.gov.ar/bitstream/handle/11336/153854/CONICET_Digital_Nro.605cad6a-fcaa-4ace-b269-957e54ea4629_A.pdf?sequence=2&isAllowed=y}

    \bibitem{ref9}
    An Internet of Things (IOT) based Smart Parking Routing System for Smart Cities. Disponible: 
    \url{https://www.academia.edu/41524293/An_Internet_of_Things_IOT_based_Smart_Parking_Routing_System_for_Smart_Cities}

    \bibitem{ref10}
    El estado de la red. Banco Interamericano de Desarrollo (BID). Disponible: 
    \url{https://conexionintal.iadb.org/2017/03/08/el-estado-de-la-red/}
    
    \bibitem{ref11}
    Wi-Fi is an essential IoT enabler. Disponible:
    \url{https://www.wi-fi.org/discover-wi-fi/internet-things}

    \bibitem{ref13}
    ¿Qué es el Internet de las cosas (IoT)?. Disponible:
    \url{https://www.redhat.com/es/topics/internet-of-things/what-is-iot}

    \bibitem{ref14}
    ¿Qué es Wi-Fi?. Disponible:
    \url{https://www.cisco.com/c/es_mx/products/wireless/what-is-wifi.html}

    \bibitem{ref15}
    What Is IoT Connectivity: A Comparison Guide. Disponible:
    \url{https://www.emnify.com/blog/iot-connectivity}

    \bibitem{ref51}
    “SigFox”. \textit{Low power wide area network (LPWAN) dedicated to Massive IoT} Disponible: \url{https://www.sigfox.com/}

    \bibitem{ref52}
    “SigFox”. \textit{Que es?; Introducción; Alcance y Desarrollo} Disponible: \url{https://sigfox.com.py/que-es-sigfox/}

    \bibitem{ref53}
    “BLE”. \textit{Bluetooth Low Energy} Disponible: \url{https://en.wikipedia.org/wiki/Bluetooth_Low_Energy}

    \bibitem{ref16}
    Comparision of LoRa and NB-IoT in Terms of Connectivity\\
    %\url{https://www.diva-portal.org/smash/get/diva2:1453316/FULLTEXT01.pdf}

    \bibitem{ref17}
    Implementación de un GateWay para conectar sensores con transceiver de LoRa a servicios públicos y privados\\
    %\url{https://www.redhat.com/es/topics/internet-of-things/what-is-iot}

    \bibitem{ref18}
    Study of LoRaWAN device and gateway setups. Disponible:
    \url{https://www.diva-portal.org/smash/get/diva2:1679773/FULLTEXT01.pdf}

    \bibitem{ref19}
    M. O. Farooq y D. Pesch, \textit{“Analyzing lora: A use case perspective,” in 2018 IEEE 4th World Forum on
    Internet of Things (WF-IoT). IEEE, 2018, pp. 355\-360.}\\
    %\url{https://www.redhat.com/es/topics/internet-of-things/what-is-iot}

    \bibitem{ref20}
    S. Daud, T. S. Yang, M. A. Romli, Z. A. Ahmad, N. Mahrom, y R. A. A. Raof,\textit{“Performance Evaluation of Low Cost LoRa Modules in IoT Applications,” IOP Conference Series:
    Materials Science and Engineering, vol. 318, p. 012053, Mar. 2018}. Disponible: 
    \url{https://iopscience.iop.org/article/10.1088/1757-899X/318/1/012053}    

    \bibitem{ref21}
    H. Boyes, B. Hallaq, J. Cunningham, y T. Watson, \textit{“The industrial internet of things (IIoT): An
    analysis framework,” Computers in Industry, vol. 101}. Disponible: 
    \url{https://linkinghub.elsevier.com/retrieve/pii/S0166361517307285}  
    

    \bibitem{ref22}
    “WIFI LoRa 32 (V2).” \url{https://heltec.org/project/wifi-lora-32}

    \bibitem{ref45}
    “ESP32 Datasheet.” \url{https://www.espressif.com/sites/default/files/documentation/esp32_datasheet_en.pdf}

    \bibitem{ref46}
    “Espressif.” \url{https://www.espressif.com/en/products/socs/esp32}

    \bibitem{ref44}
    "ESP32 the Internet of Things with ESP32". Disponible: \url{http://esp32.net/#Development}

    \bibitem{ref23}
    “F. R. Pi, “Raspberry Pi 3 Model B+” \url{https://www.raspberrypi.org/products/raspberry-pi-3-model-b}
    
    \bibitem{ref24}
    LoRa-Alliance, “About LoRaWAN® | LoRa Alliance®,” 2020. \url{https://lora-alliance.org/about-lorawan}

    \bibitem{ref25}
    V. Cola, “ESP32SingleChannelGateway,”. Disponible: \url{https://github.com/vpcola/ESP32SingleChannelGateway}

    \bibitem{ref26}
    “SX1278 | 137MHz to 525MHz Long Range Low Power Transceiver | Semtech.” Disponible: \url{https://www.semtech.com/products/wireless-rf/lora-transceivers/sx1278}

    \bibitem{ref27}
    C. L. Carrión, “Evaluación del rango de transmisión de LoRa para redes de sensores inalámbricos con LoRaWAN en ambientes urbanos,” Tesis de grado, Universidad de Cuenca, Cuenca, Ecuador, 2018

    \bibitem{ref28}
    L. Tessaro, C. Raffaldi, M. Rossi, y D. Brunelli, “LoRa Performance in Short Range Industrial Applications,” in 2018 International Symposium on Power Electronics, Electrical Drives, Automation and Motion (SPEEDAM)

    \bibitem{ref29}
    “ZigBee”. Disponible: \url{https://zigbee.org/}

    \bibitem{ref31}
    “LoraDesignGuide\_std.pdf.” Disponible: \url{https://www.semtech.com/products/wireless-rf/lora-connect/sx1276}

    \bibitem{ref32}
    “HMC5883L Triple Axis Magnetometer Breakout”. Disponible: \url{http://www.geeetech.com/wiki/index.php/HMC5883L_Triple_Axis_Magnetometer_Breakout}

    \bibitem{ref33}
    “HMC5883L Magnetometer datasheet”. Disponible: \url{https://www.farnell.com/datasheets/1683374.pdf}

    \bibitem{ref34}
    “HC-SR04 Ultrasonic Sensor Module User Guide”. Disponible: \url{https://www.handsontec.com/dataspecs/HC-SR04-Ultrasonic.pdf}

    \bibitem{ref35}
    “VL53L0X Datasheet”. Disponible: \url{https://pdf1.alldatasheet.com/datasheet-pdf/view/948120/STMICROELECTRONICS/VL53L0X.html}

    \bibitem{ref36}
    “Flutter Dev/Docs”. \textit{Internationalizing Flutter apps, Adding assets and images, state management, etc} Disponible: \url{https://flutter.dev/}

    \bibitem{ref37}
    “Cloud computing: Issues and challenges”. Disponible: \url{https://doi.org/10.1109/AINA.2010.187}

    \bibitem{ref38}
    “J. Peng, X. Zhang, Z. Lei, B. Zhang, W. Zhang, and Q. Li, Comparison of several cloud computing platforms”. Disponible: \url{https://doi.org/10.1109/ISISE.2009.94}

    \bibitem{ref39}
    “Cloud computing: A perspective study”. Disponible: \url{http://dblp.uni-trier.de/db/journals/ngc/ngc28.html#WangLYHKTF10}

    \bibitem{ref40}
    “Cloud Computing: Comparison and Analysis of Cloud Service Providers-AWs, Microsoft and Google”. Disponible: \url{https://ieeexplore.ieee.org/document/9337100}

    \bibitem{ref41}
    “Cloud computing service providers: A comparative study”. Disponible: \url{https://www.iasj.net/iasj/download/280f00e221c1a458}

    \bibitem{ref42}
    “Cloud Computing Architecture”. Disponible: \url{https://binalkagathara.medium.com/comparative-study-of-cloud-computing-and-mobile-cloud-computing-1878ae6a2d94}

    \bibitem{ref43}
    “Cloud Computing: Architecture, Services, Deployment Models, Storage, Benefits and Challenges”. Disponible: \url{https://www.researchgate.net/publication/341788106_Cloud_Computing_Architecture_Services_Deployment_Models_Storage_Benefits_and_Challenges}

    \bibitem{ref48}
    “H3: Uber\'s Hexagonal Hierarchical Spatial Index”. Disponible: \url{https://www.uber.com/blog/h3/}

    \bibitem{ref49}
    “H3: A Hexagonal Hierarchical Geospatial Indexing System”. Disponible: \url{https://github.com/uber/h3}

    \bibitem{ref50}
    “S2 Geometry”. \textit{Library for spherical geometry that aims to have the same robustness, flexibility, and performance as the very best planar geometry libraries}. Disponible: \url{https://s2geometry.io/}

    \bibitem{ref55}
    “AWS”. \textit{Amazon Web Services}. Disponible: \url{https://docs.aws.amazon.com/}

    \bibitem{ref56}
    “DrawIO”. \textit{Diagrama software}. Disponible: \url{https://app.diagrams.net/}

    \bibitem{ref57}
    “Which types of batteries for your IoT devices?”. Disponible: \url{https://www.saft.com/energizing-iot/types-batteries-iot-devices}

    \bibitem{ref58}
    “Battery-less IoT Devices”. Disponible: \url{http://www.diva-portal.org/smash/get/diva2:1651573/FULLTEXT01.pdf}

    \bibitem{ref59}
    “Nikesh Man Shakya. Design and development of energy-efficient transmission for wireless IoT modules. Networking and Internet Architecture [cs.NI]. Université Paris-Saclay, 2019. English. ffNNT :
    2019SACLL001ff. fftel-02071262v2f”.

    \bibitem{ref60}
    “Why Flutter is the most popular cross-platform mobile SDK”. Disponible: \url{https://stackoverflow.blog/2022/02/21/why-flutter-is-the-most-popular-cross-platform-mobile-sdk/}

    \bibitem{ref61}
    “Tech Stacks to Launch Your App Development into Stratosphere”. Disponible: \url{https://appinventiv.com/blog/technology-for-mobile-app-development/}

    \bibitem{ref62}
    “Arquitectura Flutter”. Disponible: \url{https://devjaime.medium.com/d%C3%ADa-3-arquitectura-flutter-y-configuraci%C3%B3n-del-proyecto-9307aa98b4fe}

\end{references}

