\section*{Resumen}
El objetivo principal de este proyecto es desarrollar un sistema de estacionamiento inteligente, que combina la tecnología de sensores y la potencia de cálculo en la nube y en el borde (edge computing). Se ha realizado un exhaustivo estudio del estado del arte en esta área, investigando los diferentes tipos de sensores disponibles y su aplicabilidad en entornos de estacionamiento.


Se llevaron a cabo pruebas de campo para evaluar la precisión y confiabilidad de los sensores en la detección de la disponibilidad de espacios de estacionamiento. Estas pruebas permitieron recopilar datos sobre la ocupación de los espacios en tiempo real y evaluar el rendimiento de los sensores en condiciones reales.


Además, se diseñó y desarrolló una aplicación móvil utilizando Flutter, un framework de desarrollo multiplataforma, que permite a los usuarios acceder a la información en tiempo real sobre la disponibilidad de espacios de estacionamiento. La aplicación ofrece funcionalidades como la búsqueda de estacionamiento cercano y la reserva de una plaza en el mismo.


Para garantizar un rendimiento óptimo y una respuesta rápida, se aprovechó la potencia de cálculo tanto en la nube como la computación de borde para el procesamiento de los datos de los sensores. Utilizando una Raspberry Pi se pudieron procesar de una manera eficiente los datos enviados por los sensores a través de LoRa y proporcionar actualizaciones en tiempo real a los usuarios de la aplicación.


En resumen, este proyecto abarcó desde el análisis del estado del arte hasta la construcción de un sistema de estacionamiento inteligente (prototipo). Se investigaron los sensores adecuados, se realizaron pruebas de campo, se desarrolló una aplicación móvil y se implementaron soluciones de computación en la nube y en el edge para ofrecer una experiencia completa al usuario. Finalmente se desarrolló un apartado de posibles mejoras al mismo.
