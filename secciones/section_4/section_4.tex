\newpage
\section{Análisis de la solución}
Uno de los objetivos principales planteados en este proyecto final, es el desarrollo de un nodo que sea capaz de detectar la ocupación de una plaza de estacionamiento para informarlo a los conductores a través de una aplicación móvil, con el objetivo de que logren poder crear una reserva. Para lograr esto en resumen se utilizaron las ventajas ofrecidas por el concepto de arquitectura de sistemas IoT (Internet of Things), el paradigma de Edge Computing (Computación de Borde) y el poder de la nube (Cloud Computing) como en torno de ejecución. A continuación se deja una representación simple de la arquitectura de la misma.

\insertimage[\label{architecture_general_diagram}]{secciones/section_4/images/arquitectura_general}{width=17cm}{Arquitectura general del sistema}

Del diagrama anterior y para una fácil comprensión del lector, desglosaremos el desarrollo de la solución en los siguientes componentes:

\begin{itemize}
    \item Nodo sensor
    \item Nodo concentrador
    \item Sistema de Reservas (servidor)
    \item Aplicación Móvil
\end{itemize}

En resumen, el sistema realiza la interconexión y comunicación entre los componentes físicos del mismo (sensores) y los componentes computacionales, que logran conectar al usuario final con el de una plaza de estacionamiento.

%Y en la imagen a continuación se presenta como sería el caso de uso real:
%\insertimage[\label{representacion_estacionamiento_real}]{secciones/section_4/images/%parking_lot}{width=17cm}{Representación Estacionamiento con sensores}

% Insertar Nodo Sensor
\subsubsection{Nodo sensor}
El nodo sensor de nuestro proyecto está compuesto de las siguientes etapas:

\begin{itemize}
    \item Etapa de adquisición de datos
    \item Etapa de procesamiento
    \item Etapa de comunicación
\end{itemize}

Como se puede ver en la figura a continuación:

\insertimage[\label{architecture_nodo_sensor}]{secciones/section_4/images/sensor}{width=18cm}{Arquitectura del nodo sensor}

Al momento de la elección de un microcontrolador para llevar a cabo nuestro prototipo, nos enfocamos en la etapa de comunicaciones principalmente. De capítulos anteriores hemos optado por LoRa como la tecnología a utilizar en este proyecto, por lo que en nuestra búsqueda, encontramos que el hardware que presentaba la mejor relación costo-beneficio para un prototipo eran las placas WiFi LoRa 32 V2 de la compañía HELTEC.

El equipo Heltec WiFi LoRa 32 es un dispositivo orientado a IoT, diseñado y producido por Heltec Automation. La tarjeta de desarrollo basa su funcionamiento en el microcontrolador ESP32, y se vale del chip integrado SX1278, para las funciones de comunicación LoRa, estos dos dispositivos interactúan por medio de una interfaz Serial Peripheral Interface (SPI).

\insertimage[\label{heltec_lora}]{secciones/section_4/images/heltec}{width=13cm}{Placa de desarrollo Heltec Lora WiFi v2}

Luego de la elección del kit de desarrollo, nos enfocamos en la determinación del sensor, y como hemos visto de la sección de “ensayos y conclusiones”, optamos por utilizar dos sensores en conjunto, un magnetómetro y otro de tipo radar. A raíz de lo disponible en el mercado, concluimos que los sensores que mejor se ajustan a nuestro proyecto son: HMC5883L (magnetómetro) y VL53L0X (radar).

De esta forma queda determinado cuál va a ser el hardware de nuestro proyecto.


\subsubsubsection{Hardware}
A continuación se muestran las características técnicas de la placa WiFi LoRa 32 V2 utilizada en este proyecto, junto con los parámetros principales de los sensores MC5883L y VL53L0X.

\enabletablerowcolor[2] % Activa el color de celda
\begin{table}[ht]
    \centering
    \caption{Tabla de parámetros Heltect WiFi LoRa 32 v2}
    \begin{tabular}{|p{2cm}|*{3}{>{\raggedright\arraybackslash}p{14cm}|}}
        \hline
        \textbf{Parámetro} & \textbf{Descripción} \\
        \hline
        Master Chip & ESP32 (240MHz Tensilica LX6 dual-core+1 ULP, 600 DMIPS) \\
        LoRa Chipset & SX1276/SX1278 \\
        Wi-Fi & 802.11 b/g/n (802.11n up to 150 Mbps) \\
        Bluetooth &Bluetooth V4.2 BR/EDR and Bluetooth LE specification \\
        Hardware Resource & UART x 3; SPI x 2; I2C x 2; I2S x 1; 12-bits ADC input x 18; 8\-bits DAC output x 2; GPIO x 22; GPI x 6 \\
        Memory &8MB(64M-bits) SPI FLASH; 520 KB internal SRAM \\
        Interface &Micro USB x 1; LoRa Antenna interface(IPEX) x 1 \\
        Dimensions &51 x 25.5 x 10.6 mm \\
        \hline
        \end{tabular}
    \label{tab:tabla_parametros_heltec}
\end{table}
\disabletablerowcolor % Desactiva el color de celda

\enabletablerowcolor[2] % Activa el color de celda
\begin{table}[ht]
    \centering
    \caption{Tabla de parámetros sensor magnético HMC5883L}
    \begin{tabular}{|p{6cm}|*{2}{>{\raggedright\arraybackslash}p{5cm}|}}
        \hline
        \textbf{Parámetro} & \textbf{HMC5883L} \\
        \hline
            Voltage Supply (Vs) &2V16 - 3V6 \\
            Digital Supply (VDDIO) (max) & 1.71V - 3V7 \\
            Abs. Max VDD/VDD IO & -0.3V - 4.8V \\
            Interface &I2C \\
            I2C Address (R,W) [RW] &0x3D, 0x3C \\
            I2C rates (kHz) & 100, 400, 3400 \\
            Resolution (ADC) & 12  bits \\
            Max Gauss (survival) & Not specified. \\
            Gauss Resolution & ±2mG - ±8G \\
            Acquisition time &6ms \\
            Active current (7Hz,10Hz) &100uA \\
            Active current &Not specified. \\
            Peak Active current &Not Specified[3] \\
            Standby mode (leakage) & 2uA \\
            Operating temperature & -30°C - 85°C \\
        \hline
        \end{tabular}
    \label{tab:tabla_parametros_sensor_magnetico}
\end{table}

\enabletablerowcolor[2] % Activa el color de celda
\begin{table}[H]
    \centering
    \caption{Tabla de parámetros sensor laser VL53L0X}
    \begin{tabular}{|p{6cm}|*{2}{>{\raggedright\arraybackslash}p{5cm}|}}
        \hline
        \textbf{Parámetro} & \textbf{VL53L0X} \\
        \hline
            Ranging chip &VL53L0X \\
            Measuring distance & 2M (Max) \\
            Measuring Mode & Default, High precision, Long Distance, High Speed \\
            Infrared emission mechanish &940nm \\
            FOV & 25º \\
            Operating Voltage & 3 - 5V \\
            Operating temperature & -20°C - 80°C \\
        \hline
        \end{tabular}
    \label{tab:tabla_parametros_sensor_laser}
\end{table}


\subsubsubsection{PinOut}
A continuación se visualiza el pinout del kit de desarrollo. Más adelante incluiremos un esquemático de la conexión entre el sensor y la misma.

\insertimage[\label{heltec_pinout}]{secciones/section_4/images/pinout_heltec}{width=16cm}{Pinout placa de desarrollo Heltec Lora WiFi v2}

\subsubsubsection{Conexión}
A continuación se visualiza la conexión entre el kit de desarrollo y los sensores utilizados en este proyecto. Cabe aclarar que este es el modelo esquemático 
esta diseñado unicamente con los kits de desarrollo, más adelante incluiremos un apartado del diseño del PCB.

%\insertimage[\label{heltec_conexion_sensores}]{secciones/section_4/images/heltec_conexion_sensores}{width=13cm}{Diagrama de conexión entre Heltec y Sensores}

\insertimage[\label{heltec_diagrama_conexion}]{secciones/section_4/images/heltec_diagrama_conexion}{width=16cm}{Diagrama de conexión esquematico entre Heltec y Sensores}

\subsubsubsection{Configuración del nodo sensor}
Para poder explicar mejor el software implementado en el sensor, presentamos primero un pequeño diagrama inicial de eventos del sistema. Cabe aclarar que este es un diagrama simplificado y no incluye toda la interacción del sistema.

\insertimage[\label{diagrama_de_eventos_del_sistem}]{secciones/section_4/images/diagrama_de_eventos_del_sistem}{width=13cm}{Diagrama de eventos del sistema}

Los datos a transmitirse desde el nodo sensor hacia el colector son recopilados por la tarjeta Heltec WiFi LoRa 32. Esta tarjeta permite la interacción con el magnetómetro y el sensor de tiempo de vuelo a través del protocolo I2C.

El microprocesador ESP32 de la tarjeta Heltec, maneja el chip SX1278 (LoRa) y sus parámetros de transmisión (frecuencia, potencia, SF) así como las claves de seguridad de la red. La programación de este controlador se encuentran disponibles en el repositorio de GitHub del proyecto. Ver anexo ~\ref{codigo_sensor}

\insertimage[\label{architectura_nodo_facil}]{secciones/section_4/images/architectura_nodo_facil}{width=13cm}{Diagrama de bloques nodo sensor}

\subsubsubsection{Estimación tiempo de vida util - bateria del nodo sensor}{\label{sec:bateria_del_nodo_sensor}}
Dada la naturaleza de nuestro proyecto, uno de los aspectos clave del mismo en la implementación del nodo sensor es el diseño de este para que consuma muy poca energía y además acompañar esta implementación junto con la estimación en el tiempo de vida útil del nodo.
Al lograr estimar la vida útil de la batería, se pueden tomar mejores decisiones para realizar una planificación del mantenimiento y la sustitución anticipada de la batería, lo que contribuye a evitar interrupciones en el funcionamiento del sistema.

A continuación introduciremos los conceptos claves junto con los cálculos necesarios para lograr la implementación y la estimación del tiempo de vida.

\subsubsubsectionanumnoi{ESP8266 - Low-Power Management}
Conforme se destacó en la introducción de esta sección, resulta crucial considerar el consumo de energía durante el desarrollo del nodo sensor. Nuestra investigación revela que el microcontrolador a emplear ofrece cuatro (4) perfiles de energía distintos, como se muestra en la imagen adjunta.

\insertimage[\label{esp_low_power}]{secciones/section_4/images/esp_low_power}{width=17cm}{Modos de consumo de energía ESP8266}

Durante el modo de suspensión profunda (deep sleep), la CPU principal se apaga, mientras que el Coprocesador Ultra-Bajo Consumo (ULP) puede tomar lecturas de sensores y despertar a la CPU según sea necesario. Este tipo de perfil de energía es útil para diseñar aplicaciones en las que la CPU debe despertarse por un evento externo, un temporizador o una combinación de estos eventos, manteniendo un consumo mínimo de energía.

Debido a que la memoria RTC se mantiene activa, su contenido se conserva incluso durante el modo de suspensión profunda y se puede recuperar una vez que el chip se despierta. Para una mejor ejemplificación adjuntamos el diagrama de bloques con aquellos bloques que siguen activos durante el modo de funcionamiento Deep-Sleep.

\insertimage[\label{diagrama_low_power_bloques}]{secciones/section_4/images/diagrama_low_power_bloques}{width=15cm}{Bloques activos en modo Deep-Sleep}

\subsubsubsectionanumnoi{Funcionamiento del nodo (considerando los modos de energía)}
En apartados anteriores, hemos explicado el principio de funcionamiento de nuestro nodo sensor y los distintos modos de consumo de energía que ofrece el microcontrolador ESP8266. Ahora, detallaremos cómo funciona nuestro nodo en conjunto con los periféricos y el software diseñado.

Antes de adentrarnos, asumiremos que el nodo ya está configurado, es decir, las variables necesarias para una comunicación exitosa con la etapa de comunicaciones LoRa han sido establecidas (frecuencia, canales de TX y RX, SF, etc.). Además, los pines para la comunicación mediante I2C han sido configurados previamente. En esta ocasión, nos centraremos en el funcionamiento del nodo para llevar a cabo la detección de vehículos.

\insertimage[\label{power_modes_system}]{secciones/section_4/images/power_modes_system}{width=15cm}{Esquema simplificado de los modos de energia y su interacción.}

Tras la configuración previa, el microcontrolador entra en modo de Deep-Sleep. Durante este tiempo, toma lecturas del magnetómetro HMC5883L, procesa los datos y determina si se ha superado el valor límite establecido. Es importante mencionar que durante esta fase, el microcontrolador utiliza el coprocesador ULP para el cómputo de los datos, lo que permite reducir el consumo de energía a tan solo 10 μA durante los 500 mS que el procesador principal se encuentra apagado, según las especificaciones oficiales.

\insertimage[\label{esquema_energia}]{secciones/section_4/images/esquema_energia}{width=15cm}{Esquema de energía interno ESP8266}

Cuando se detecta que el valor límite ha sido superado, el coprocesador llama a una función previamente definida, indicando a la ESP que debe salir del modo Deep-Sleep y entrar en modo de funcionamiento activo.

En el modo activo, el microcontrolador solicita la lectura del sensor láser VL53I0x y espera una interrupción que se produce cuando la lectura está lista. Una vez que se obtiene la lectura, se compara con la última registrada para determinar si ha habido algún cambio significativo. En caso afirmativo, se envía el valor a través de LoRa.
Mientras se espera recibir una señal de ACK (acknowledgement) por parte del nodo de borde, el microcontrolador entra en un estado de sueño intermitente durante un segundo para evitar un gasto innecesario de energía.

Finalmente, una vez que se recibe la señal de ACK, el microcontrolador vuelve a ingresar en el modo de Deep-Sleep, preparado para realizar nuevas detecciones.

\subsubsubsectionanumnoi{Cálculo teórico - tiempo de vida batería}
Para poder estimar de manera precisa el tiempo de vida de la batería que se requiere para nuestro equipo, se ha realizado una cuidadosa identificación de dos patrones de consumo claramente definidos, a los cuales hemos denominado \textit{Alto Consumo} y \textit{Bajo Consumo}, cada uno con sus respectivos periodos de funcionamiento "ON" y de inactividad "OFF". Este desglose minucioso nos proporcionará una estimación más detallada y precisa del tiempo de vida esperado para la batería, permitiéndonos tomar decisiones informadas acerca de la capacidad y durabilidad óptimas para el sistema electrónico en cuestión. Este análisis está basado en que muchos de los sistemas IoT pasan una pequeña parte de su tiempo en un modo activo (o de funcionamiento) y el resto en modo de inactividad.

Para comenzar con el cálculo teórico, vamos a tomar los niveles de consumo de acuerdo a los datasheet de los fabricantes. En el sistema de bajo consumo tenemos los siguientes parámetros:

\insertindexequation[\label{consumo_ulp_modo_activo}]{I_{ULP A} = 0.0004 mA}{consumo ULP en modo activo}
\insertindexequation[\label{consumo_ulp_modo_off}]{I_{ULP Off} = 0.00001 mA}{consumo ULP en modo off}
\insertindexequation[\label{tiempo_ulp_modo_activo}]{t_{ULP A} = 2 ms}{tiempo en modo activo}
\insertindexequation[\label{tiempo_ulp_modo_off}]{t_{ULP Off} = 500 ms}{tiempo en modo off}

Por lo tanto la corriente total queda definida como:
\insertindexequation[\label{corriente_total}]{I_{ULP} = \frac{(I_{ULP A}* t_{ULP A} + I_{ULP Off} * t_{ULP Off})}{t_{ULP A} + t_{ULP Off}}}{Corriente total}{}

\insertindexequation[\label{corriente_magnetometro}]{I_{Magnetometer}= 0.00002 mA}{Corriente total}{}

Finalmente el valor de la corriente de bajo consumo queda definida como la suma del procesador ULP y la corriente consumida por el magnetómetro.

\insertindexequation[\label{corriente_I_low_power}]{I_{Low Power}= I_{ULP} + I_{Magnetometer}}{Corriente total del sistema bajo consumo}{}

En el sistema de alto consumo consideramos los siguientes valores:

\insertindexequation[\label{corriente_I_low_power}]{t_a}{medido en ms}{representa el tiempo que permanece cada auto en un slot}
\insertindexequation[\label{corriente_esp_modo_activo}]{I_{ESP A} = 25 mA}{consumo ESP en modo activo}{}
\insertindexequation[\label{corriente_ulp_modo_off}]{I_{Laser ON} = 19 mA}{consumo ULP en modo off}{}
\insertindexequation[\label{tiempo_esp_modo_activo}]{t_{ESP A} = 300 ms}{tiempo que la ESP permanece en modo activo}{}
\insertindexequation[\label{tiempo_laser_activo}]{t_{Laser On} = 40 ms}{tiempo que el laser está activo}{}

Con las variables anteriores queda definido el tiempo en la que el nodo sensor se encuentra recolectando la información sobre la ocupación de un slot de estacionamiento.

\insertindexequation[\label{corriente_esp_collecting}]{I_{ESP Collecting} = \frac{I_{ESP A} * t_{ESP A}}{t_{ESP A} + t_a}}{corriente de la ESP mientras collecta información}{}
\insertindexequation[\label{tiempo_laser_activo}]{I_{Laser Collecting} = \frac{I_{Laser ON} * t_{Laser ON}}{t_{Laser ON} + t_a}}{corriente total mientras el laser collecta información}{}
\insertindexequation[\label{tiempo_laser_activo}]{I_{Collecting} = I_{ESP Collecting} + I_{Laser Collecting} }{corriente total consumida mientras se collecta información}{}

Por último, una vez recolectada la información, tenemos la etapa de envío y recepción de información.

\insertequation{I_{ESP Light Sleep} =0.8 mA}
\insertequation{I_{sx stby}=0.0000015 mA}
\insertequation{I_{sx tx} = 100 mA}
\insertequation{I_{sx rx} = 12 mA}

\insertequation{t_{sx tx} = 175 ms}
\insertequation{t_{sx rx} = 1000 ms}
\insertequation{t_{sx stby} = 27 ms}
\insertequation{t_{ESP A} = t_{sx stby}}
\insertequation{t_{ESP_Light_Sleep} = t_{sx tx} + t_{sx rx}}

La etapa de comunicación queda definida entonces por la suma entre la corriente que consume el nodo mientras transmite y mientras recibe como se puede ver en las fórmulas a continuación.

\insertequation{I_{Transmitting\ ESP} = \frac{I_{ESP\ A} * t_{ESP\ A} + I_{ESP\ Light\ Sleep} * t_{ESP\ Light\ Sleep}}{t_a+ t_{ESP\ A}+ t_{ESP\ Light\ Sleep}}}

\insertequation{I_{Transmitting\ SX} = \frac{t_a *I_{sx\ stby} + t_{sx\ stby} * I_{sx\ stby} +  t_{sx\ rx} * I_{sx\ rx} + t_{sx\ tx} * I_{sx\ tx}}{t_a+ t_{sx\ stby} +  t_{sx\ rx} + t_{sx\ tx}}}

\insertequation{I_{Transmitting} = I_{Transmitting\ ESP} + I_{Transmitting\ SX}}
\insertequation{I_{High\ Power} = I_{Collecting} + I_{Transmitting}}

En este punto ya tenemos todos los valores de intensidad expresados y podemos estimar el tiempo de vida útil de una batería. Asumiendo que tenemos una batería de 5000 mAh, en la gráfica a continuación se puede visualizar el tiempo de vida estimado en función de la permanencia de vehículos en una plaza del estacionamiento.

\insertimage[\label{calculo_bateria}]{secciones/section_4/images/calculo_bateria}{width=17cm}{Calculo teórico tiempo de vida de la batería}

En la sección de anexos, puede encontrarse el código para el cálculo y generación de la gráfica anterior. Ver anexo ~\ref{codigo_estimacion_bateria}

% Insertar Nodo de Borde
\clearpage
\subsection{Nodo Concentrador}
Este nodo desempeña el papel de intermediario entre los nodos sensores y el servidor. En una de sus funciones, recopila datos de los nodos sensores utilizando la tecnología LoRa y, en la otra, los transmite al servidor a través del protocolo HTTP. Su funcionamiento puede verse resumido en la figura a continuación.

\insertimage[\label{nodo_concentrador_funcionamiento}]{secciones/section_4/images/nodo_concentrador_overview}{width=14cm}{Diagrama simplificado de funcionamiento nodo concentrador}



\subsubsection{Hardware}
El concentrador debe estar habilitado para recibir mensajes a través de la tecnología LoRa, lo cual implica la incorporación de un transceptor SX1278 en su diseño. Asimismo, es esencial que el dispositivo sea capaz de efectuar solicitudes HTTP, lo que implica que debe contar con acceso a la red de Internet. En cuanto a la modalidad de conexión, se ha optado por la conectividad inalámbrica mediante Wi-Fi, aunque cabe señalar que también se podría implementar una conexión cableada si se considera más apropiada en circunstancias específicas.

Dadas las limitaciones previamente mencionadas, a pesar de que es técnicamente factible construir el concentrador utilizando la misma placa de desarrollo que se emplea en el nodo sensor, se ha optado por utilizar una Raspberry Pi. Esta elección se fundamenta en su capacidad para cumplir con los requisitos establecidos, la ausencia de restricciones energéticas, la simplificación que aporta al proceso de desarrollo del software, y la conveniencia de contar con una unidad disponible para ser usada.

\insertimage[\label{raspberry_pi}]{secciones/section_4/images/raspberry_pi}{width=12cm}{Raspberry Pi 3 Model B+}

La Raspberry Pi es una computadora de placa unica (SBC por las siglas en ingles Single Board Computer), utiliza una arquitectura de procesador ARM distinta a la que estamos acostumbrados a utilizar en nuestros ordenadores. Esta arquitectura es de tipo RISC (Reduced Instruction Set Computer), es decir, utiliza un sistema set de instrucciones realmente simple lo que le permite ejecutar tareas con un mínimo consumo de energía y cuenta con salidas GPIO (General Purpose Input/Output - Entrada/Salida de Propósito General) lo que nos permite a través de la implementación del protocolo SPI poder dotar a la placa de comunicación LoRa.

Teniendo en cuenta lo mencionado anteriormente, en el diagrama de bloques a continuación se visualizan los módulos actuantes para el nodo concentrador.

\insertimage[\label{concentrador_hardware}]{secciones/section_4/images/concentrador_hardware}{width=16cm}{Diagrama de bloques conexion nodo concentrador}


\subsubsubsection{PinOut}
A continuación se visualiza el pinout de la placa Raspberry Pi 3 Model B+:

\insertimage[\label{raspberry_pi_pinout}]{secciones/section_4/images/raspberry_pi_pinout}{width=16cm}{Pin Out - Raspberry Pi 3 Model B+}


\subsubsubsection{Plano esquemático}
En el diagrama subsecuente se presentan de manera detallada las interconexiones que existen entre el módulo LoRa SX1278 y la Raspberry Pi, establecidas a través de los pines que componen la interfaz de comunicación serial SPI. 

\insertimage[\label{concentrator_schematic}]{secciones/section_4/images/concentrator_schematic}{width=17cm}{Conexión entre la Raspberry Pi y el módulo LoRa}



\subsubsection{Software}
El nodo concentrador desempeña dos funciones primordiales: la adquisición de información procedente de sensores mediante la tecnología de comunicación LoRa, seguida de la transmisión subsiguiente de estos datos utilizando el protocolo HTTP.
La idea es desarrollar un sistema que, al ser activado inicialmente, tenga la capacidad de autoconfigurarse o de iniciar un proceso que le permita obtener los parámetros necesarios del usuario. Una vez que se haya llevado a cabo la configuración, el sistema procederá a iniciar la ejecución del programa principal (Figura \ref{concentrador_flujo_general}).

\insertimage[\label{concentrador_flujo_general}]{secciones/section_4/images/concentrador_flujo_general}{width=7cm}{Flujo del concentrador}


\subsubsubsection{Pila de tecnologías}
El software del concentrador esta contruidro sobre la base de una pila de tecnologias que puede resumirse a las que conforman la figura \ref{concentrador_capas}

\insertimage[\label{concentrador_capas}]{secciones/section_4/images/concentrador_capas}{width=10cm}{Capas del concentrador}

Empezando desde abajo hacia arriba el primer elemento es Raspberry Pi OS, anteriormente conocido como Raspbian, representa una distribución del sistema operativo GNU/Linux, construida sobre la base de Debian y especialmente diseñada para la utilización en la placa computadora de bajo costo Raspberry Pi. Esta plataforma vio la luz por primera vez en junio de 2012 y desde el año 2015 ha sido oficialmente respaldada por la Raspberry Pi Foundation como el sistema operativo predeterminado para su gama de placas SBC (Single Board Computer).
Es importante destacar que Raspbian, un proyecto concebido de manera independiente por Mike Thompson y Peter Green, sigue evolucionando activamente. Se caracteriza por su notable optimización para el hardware de Raspberry Pi. Desde una perspectiva técnica, este sistema operativo constituye una adaptación no oficial de Debian armhf, adaptada para el procesador (CPU) de Raspberry Pi. Además, incorpora un soporte altamente optimizado para operaciones de coma flotante por hardware, lo que resulta en un rendimiento mejorado en determinados escenarios. Esta adaptación fue esencial debido a la inexistencia de una versión oficial de Debian armhf compatible con la arquitectura ARMv6, presente en la Raspberry Pi.
Raspberry Pi OS se elige como el sistema operativo para las Raspberry Pi debido a su eficiencia en el hardware y amplia compatibilidad con periféricos, lo que lo convierte en la opción ideal para nuestra aplicacion.

El segundo elemento, Python, es un lenguaje de programación de alto nivel, multipropósito e interpretado, que se destaca por su sintaxis clara y legible. Fue creado por Guido van Rossum y lanzado por primera vez en 1991, y desde entonces ha ganado una amplia aceptación en la comunidad de desarrollo de software debido a su facilidad de uso y versatilidad. Python ofrece una amplia biblioteca estándar que abarca una variedad de áreas, lo que facilita la implementación de una amplia gama de aplicaciones y proyectos. Su enfoque en la legibilidad del código fomenta la escritura de programas limpios y mantenibles, lo que lo convierte en una elección popular tanto para principiantes como para desarrolladores experimentados en diversas disciplinas, desde la ciencia de datos hasta el desarrollo web.
La elección de Python como lenguaje de programación para el desarrollo del concentradorse justifica por varias razones fundamentales. En primer lugar, Python es altamente compatible con Raspberry Pi OS. Esto garantiza una integración fluida y una experiencia de desarrollo sin problemas.
Además, contamos con experiencia de desarrollo con el lenguaje Python. La familiaridad con la sintaxis y las bibliotecas de Python permite un desarrollo eficiente y una resolución efectiva de problemas.


\subsubsubsection{Configuración Inicial}
Durante la inicialización de este dispositivo, y con el objetivo de simplificar su configuración, hemos desarrollado una aplicación que se ejecuta en el proceso de arranque del nodo concentrador.

La placa sirve una interfaz web (Figura \ref{wifi_conf}), que permite llevar a cabo la configuración de la red WiFi utilizada para la comunicación con el servidor. Configurar la conexión WiFi es de suma importancia, ya que sin esta configuración, no se puede proporcionar información sobre la disponibilidad del estacionamiento.

\begin{images}[\label{wifi_conf}]{Configuración del WiFi}
    \addimage[\label{wifi_conf:bootstrap_1}]{secciones/section_4/images/bootstrap_1}{width=5.1cm}{}
    \addimage[\label{wifi_conf:bootstrap_2}]{secciones/section_4/images/bootstrap_2}{width=5.1cm}{}
    \addimage[\label{wifi_conf:bootstrap_3}]{secciones/section_4/images/bootstrap_3}{width=5.1cm}{}
    \end{images}

Una vez que se ha establecido la conexión WiFi, se realiza una consulta HTTP para descargar automáticamente las credenciales de identificación desde el servidor (Figura \ref{credentials_conf:identificacion_estacionamiento}). Estas credenciales validan al nodo concentrador y al estacionamiento dentro del sistema.

Luego, debemos guardar los cambios, y la aplicación indicará que el nodo está listo para operar (Figuras \ref{credentials_conf:nodo_concentrador_listo_1} y \ref{credentials_conf:nodo_concentrador_listo_2}). En este punto, el nodo está preparado para recibir y procesar los datos obtenidos por los sensores.


\begin{images}[\label{credentials_conf}]{Configuración de las credenciales}
\addimage[\label{credentials_conf:identificacion_estacionamiento}]{secciones/section_4/images/bootstrap_4}{width=5.1cm}{} 
\addimage[\label{credentials_conf:nodo_concentrador_listo_1}]{secciones/section_4/images/bootstrap_5}{width=5.1cm}{}
\addimage[\label{credentials_conf:nodo_concentrador_listo_2}]{secciones/section_4/images/bootstrap_6}{width=5.1cm}{}
\end{images}


\subsubsubsection{Concentrador}
El bucle principal del concentrador se compone de dos elementos fundamentales. En primer lugar, encontramos la fase encargada de recibir los mensajes provenientes de los sensores a través de la tecnología LoRa. Esta fase implica la deserialización de los datos, su validación y la confirmación de su recepción mediante el envío de un acuse de recibo (ACK) de vuelta al nodo sensor correspondiente. Por otro lado, existe un módulo específico encargado de procesar estos mensajes y transmitirlos al servidor central.
La interconexión de estos dos sistemas se realiza mediante una cola, lo que habilita al concentrador para gestionar una mayor cantidad de mensajes entrantes en momentos de alta demanda. Esto posibilita la postergación de la retransmisión de los datos al servidor para un momento más oportuno. Es importante destacar que, si bien esta estrategia puede resultar en notificaciones con cierto retraso para los usuarios, se evita la pérdida de mensajes enviados por los sensores. Este enfoque, en última instancia, contribuye a prolongar la vida útil de la batería de los dispositivos al reducir la necesidad de retransmitir los datos. 

\insertimage[\label{concentrador_flujo}]{secciones/section_4/images/concentrador_flujo}{width=14cm}{Flujo del concentrador}


% Insertar seccion sofware
\clearpage

\subsection{Sistema de Reservas}
Esta sección presenta una visión general de la arquitectura propuesta para el sistema de Smart Parking, contextualizando brevemente el sistema multiagente a integrar y las razones detrás de la utilización de tecnologías en la nube. También se describe la estructura, detallando el papel de los componentes principales del sistema.

\subsubsection{Resumen de la Arquitectura}

Uno de los principales objetivos de la arquitectura propuesta es permitir la interacción del usuario con el sistema multiagente desarrollado en este proyecto. En resumen, este sistema realiza la interconexión entre componentes físicos y recursos computacionales con un enfoque basado en la descentralización del sistema. En la figura a continuación se puede observar la arquitectura completa del sistema y su interconexión con la interfaz del usuario y los sensores.

% CAMBIAR FOTO ARQUITECTURA
\insertimage[\label{arquitectura_software}]{secciones/section_4/images/sistemas_reservas}{width=16cm}{Resumen de Arquitectura en el Cloud}

Como se puede apreciar debemos poder tener una comunicación entre los sensores físicos y la información visualizada por los conductores. Pensando esto en una escalabilidad muy alta es necesario integrar todo el sistema a una plataforma que permita la comunicación entre ambas partes sin tener limitaciones que afectan la experiencia de usuario. Basado en esto hemos optado por alojar la solución en el cloud (específicamente AWS), como se ha mencionado en capítulos anteriores esta opción no depende solo de un servidor físico. Alojar la solución en el cloud permite un número ilimitado de servidores actuando como un único sistema unificado, permitiendo así la flexibilidad, escalabilidad y ahorro de costos dado que se paga únicamente por los recursos utilizados. En este caso el uso del cloud será como PaaS (Platform as a Service) que variará los recursos computacionales en función de la demanda.

En esta arquitectura, los componentes principales son autenticación del usuario, capa de base de datos, motor en tiempo real, manejo de eventos por publicación con SNS.

\subsubsubsection{Autenticación del usuario}
El objetivo principal de un módulo de autenticación es verificar la identidad del usuario, lo que implica métodos de validación de datos para confirmar su autenticidad. Es importante contar con un módulo de autenticación flexible y accesible en todo momento para que los usuarios puedan acceder a la aplicación. Dado que también es a través de este módulo que se verifican los roles del mismo, de esta forma se pueden limitar las características a la que un usuario tiene acceso frente a un administrador (no desarrollado en este trabajo).

En las aplicaciones actuales, es comúnmente compatible la autenticación mediante correo electrónico y contraseña, números de teléfono e incluso proveedores como Google, Facebook y Twitter. Cuando se utiliza uno de estos métodos, el sistema puede reconocer qué usuario está llevando a cabo una determinada acción y, en función del rol asignado, puede controlar el acceso al uso de los recursos del sistema. Para nuestro caso se decidió implementar la autenticación con un proveedor social (Social Provider) como Google dado que es ampliamente difundido y la mayor parte del mercado cuenta con un dispositivo Android lo que hace que deba tener una cuenta de Google activa.

\subsubsubsection{Base de datos NoSQL}
Ha habido un aumento masivo en el almacenamiento de datos en los últimos años, en gran parte debido a las enormes redes sociales como Facebook y Twitter, los motores de búsqueda como Google y otros sitios web que almacenan grandes cantidades de datos. Por lo tanto, surgen varios desafíos relacionados con el almacenamiento, captura, análisis y transferencia de estos datos, y ahí es donde entran en juego las bases de datos NoSQL (por ejemplo, MongoDB, Firestore, DynamoDB), conocidas por funcionar bien con big data y aplicaciones en tiempo real.

\subsubsubsectionanumnoi{Concepto}
La principal ventaja de utilizar este tipo de base de datos es su modelo extremadamente flexible. Al comparar con el desarrollo de bases de datos relacionales como MySQL o Postgres, se nota la importancia de modelar toda su estructura en un esquema predefinido, que incluye tablas, campos, restricciones y también el tipo de cada campo. En las bases de datos NoSQL, también conocidas como bases de datos no relacionales, no es necesario crear esquemas, lo que significa que no hay restricciones a nivel de base de datos en tablas predefinidas o qué tipo de datos colocar en ciertos campos, lo que permite iterar fácilmente en el diseño de la base de datos, agregando o cambiando campos según sea necesario sin el alto riesgo de corromper los procesos actuales. Sin embargo, es importante realizar cierta planificación para que la estructura no se desorganice con el tiempo.

Las bases de datos NoSQL son, en muchos casos, más rápidas y más económicas de administrar debido al aspecto de la escalabilidad. La escalabilidad es de gran importancia en la industria actual, que está creciendo cada vez más, donde los sitios web y las aplicaciones tienden a atraer a más clientes y, por lo tanto, más datos. Las bases de datos relacionales tienden a utilizar métodos de escalabilidad vertical, lo que significa que se deben agregar componentes adicionales, como memoria RAM, potencia de CPU, puertos de red y otros componentes normalmente costosos, a una máquina para escalarla. Este proceso de escalabilidad vertical tiene algunas desventajas, como el costoso costo de nuevos materiales o incluso la necesidad de una nueva infraestructura. Desde este punto de vista, el hardware y la centralización de datos en una sola máquina tienen muchas desventajas.

% CAMBIAR FOTO VERTICAL VS HORIZONTAL
\insertimage[\label{scaling_concept}]{secciones/section_4/images/scaling_concept}{width=13cm}{Comparativa escalado vertical vs horizontal}

Por otro lado, el enfoque de escalabilidad horizontal permite agregar nuevas máquinas al clúster de recursos como nodos y facilita el mantenimiento y la adición de nuevos nodos en bases de datos no relacionales. Este enfoque implica separar una parte secuencial de la lógica en partes más pequeñas para que se puedan realizar en varias máquinas en paralelo, aumentando la eficiencia de las lecturas y escrituras a gran escala.

Existen diferentes tipos de bases de datos NoSQL, como los pares de clave-valor, que son las más simples de ellas, donde solo están presentes las claves y sus respectivos valores, siendo ideales para conjuntos de datos grandes con detalles muy básicos. Son extremadamente rápidas, personalizables y no muy complicadas, similares a un hash o un arreglo asociativo. Además, existen bases de datos de gráficos, que no son tan comunes, donde todo se trata como un nodo, que puede tener relaciones con otros nodos a través de lo que se llama un "edge".

Finalmente, las bases de datos de documentos almacenan datos similares a la Notación de Objetos JavaScript (JSON), donde los datos están envueltos entre llaves y tienen pares de clave-valor. Estos valores pueden contener cadenas, números, arreglos, objetos incrustados, etc. A nivel de base de datos, es completamente sin esquema, por lo que la aplicación puede ser muy dinámica


\subsubsubsection{Indexación H3 - Uber System Approach}
En el desarrollo de nuestro proyecto nos encontramos con un potencial problema asociado a costos y performance que tiene que ver con la base de datos y cómo los datos dentro de esta podrían ser consultados. Inicialmente se consideró que teniendo como dato la Latitud y Longitud del estacionamiento junto con un radio de búsqueda era suficiente para que un usuario pueda hallar el estacionamiento próximo a su destino, sin embargo esto requeriría consultar todos los datos de la tabla cada vez que un usuario procesa una solicitud lo cual además de ser altamente costoso es también no performante.

De nuestra investigación, descubrimos que actualmente Uber (plataforma de contratación de taxis online) había tenido este problema y desarrolló un sistema de indexación geoespacial denominado H3 para lidiar con este problema.

% CAMBIAR FOTO INDEX UBER
\insertimage[\label{uber_index_globe}]{secciones/section_4/images/uber_index_globe}{width=15cm}{Representación indexación H3 Uber en imagen de globo}

Esta biblioteca (H3) es un sistema de indexación geoespacial que divide el mundo en celdas hexagonales. El mismo es de código abierto y está disponible en Github para la comunidad. H3 asigna un índice jerárquico único a cada celda, formando una malla hexagonal que cubre toda la superficie terrestre. Este enfoque jerárquico permite representar ubicaciones con diferentes niveles de precisión, desde grandes áreas geográficas hasta puntos específicos en el mapa. La estructura hexagonal se adapta particularmente bien a la representación de datos geoespaciales debido a su uniformidad y capacidad para abordar el problema de la distorsión de área, común en otros sistemas de cuadrícula.

% CAMBIAR FOTO INDEX UBER rectangulo
\insertimage[\label{hexagono_uber}]{secciones/section_4/images/hexagono_uber}{width=15cm}{Vista de una celda con codificación H3}

Una de las ventajas más destacadas del sistema H3 es su capacidad para reducir significativamente el número de consultas a una base de datos, al organizar la información en una jerarquía de celdas hexagonales. De esta forma, el sistema reduce la cantidad de datos que deben ser recuperados en cada consulta, lo que se traduce en una mejora notable en el rendimiento y la eficiencia del acceso a la base de datos. Esto es especialmente relevante cuando se trata de grandes volúmenes de datos geoespaciales que deben ser procesados en tiempo real para soportar aplicaciones como rastreo de vehículos, optimización de rutas, etc.

Por lo explicado anteriormente, es que hemos adoptado el sistema de indexado H3 en nuestro proyecto para poder representar donde se encuentran los estacionamientos. Y al momento en el que el conductor realice una consulta, de fondo se estarán convirtiendo la latitud y longitud a un hash específico lo que nos permitirá optimizar el número de consultas a realizar para encontrar la opción que desea el usuario. En la imagen a continuación se puede apreciar un mapa de la ciudad de Rosario en la cual hemos graficado como se representa el mismo utilizando el indexado de H3. En este caso, estamos utilizando la resolución por defecto que es de 8 y corresponde con un promedio de 0.737327598 KM2 de área hexagonal aproximadamente.

% CAMBIAR FOTO INDEX UBER rosario cuadricula
\insertimage[\label{rosario_celdas}]{secciones/section_4/images/rosario_celdas}{width=15cm}{Mapa de la ciudad de Rosario con índices superpuestos}

\subsubsection{Flutter como framework principal}
Flutter fue adoptado como el lenguaje principal para escribir nuestra aplicación móvil dado que es un marco de desarrollo de aplicaciones móviles de código abierto creado por Google para crear aplicaciones de alto rendimiento para iOS y Android en una única base de código. 

Esto evita tener que realizar el doble de trabajo en caso que se quiera correr en más de una plataforma. Además actualmente el mismo, también soporta integración con aplicaciones de escritorio e interfaz web, lo que en un principio podría extenderse las funcionalidades a Windows o MacOS (aunque no está cubierto en este trabajo final). Además de lo mencionado anteriormente, este framework destaca hoy en el marcado por la abstracción lograda por el mismo, ya que los desarrolladores pueden enfocarse en la lógica del negocio en sí y utilizar los widgets que la comunidad crea para la parte visual de la aplicación, logrando buenos resultados aceptables en cuanto al diseño gráfico de la aplicación y preservando todas las funcionalidades de fondo en cuanto a performance nativa. En las siguientes secciones agregaremos pantallas de la aplicación creada.

\subsubsection{Búsqueda de estacionamientos disponibles}
El proceso de búsqueda y reserva de estacionamientos disponibles comienza con el usuario ingresando a la aplicación, previamente autenticado. A través del uso del mapa o la barra de búsqueda, seleccione su destino y la ubicación cercana donde desea estacionar su vehículo. Al seleccionar el destino, la aplicación envía una consulta al backend, que a su vez devuelve una lista de estacionamientos cercanos con disponibilidad.

Para evitar sobrecargar el sistema con datos innecesarios, solo se muestra la disponibilidad de los estacionamientos en la parte gráfica. Una vez que el usuario visualiza los estacionamientos disponibles, puede seleccionar el que prefiera y proceder a confirmar su reserva. En ese momento, el backend verifica si el booking es aceptado o rechazado, informando al usuario sobre el resultado. Para una mejor visualización de lo descrito en este párrafo se adjunta a continuación un diagrama de flujos que corresponde al proceso de creación de reserva de una usuario.

% CAMBIAR FOTO BUSQUEDA DESTINO
\insertimage[\label{busqueda_parking_software}]{secciones/section_4/images/busqueda_parking_software}{width=5cm}{Diagrama de flujos con representación de la búsqueda de destinos}

En caso de que la reserva sea aceptada, el usuario puede ver el estado de su reserva en la sección "Mis reservas", manteniéndose al tanto de los detalles de la reserva realizada. Este sistema de búsqueda y reserva proporciona a los usuarios una forma eficiente y conveniente de encontrar estacionamientos disponibles en las cercanías de su destino, mejorando la experiencia de estacionamiento para los usuarios de la aplicación.

Algo a remarcar en en este proceso, es el funcionamiento del backend y como tiene un rol crucial en la búsqueda de estacionamientos cercanos. De secciones anteriores hemos visto que utilizamos el sistema de indexado H3 de Uber en nuestra base de datos y a continuación adjuntamos de una manera gráfica como es el proceso que realiza el backend para decidir el estacionamiento más cercano próximo a nuestro destino. De esta forma el lector puede interpretar de una manera gráfica y simple cómo está funcionando el sistema sin necesidad de incurrir en detalles técnicos específicos.

% CAMBIAR FOTO PROCESO DE BUSQUEDA
\insertimage[\label{funcionamiento_h3}]{secciones/section_4/images/funcionamiento_h3}{width=13cm}{Representación funcionamiento búsqueda basada en índices}

\subsubsubsection{AWS (Amazon Web Services)}

Es una plataforma de servicios en la nube que ofrece soluciones para el desarrollo y el despliegue de aplicaciones web. La elección de AWS como opción para el hosteo del backend de la aplicación del proyecto se basa en varias características: 

\begin{itemize}
    \item El equipo tiene más de 4 años de experiencia utilizando esta compañía.
    \item Son el mayor proveedor de servicios cloud del mundo con más del 70\% del mercado.
    \item Permiten la escalabilidad de servicios y gran disponibilidad asegurando un modelo de mercado de pago por demanda.
    \item Etc.
\end{itemize}

AWS, siglas de Amazon Web Services, es una plataforma de servicios en la nube que ofrece soluciones para el desarrollo y el despliegue de aplicaciones web. La elección de AWS como opción para el hosteo del backend de la aplicación se basa en varios factores que se explicarán en esta tesis. Al usar AWS, se puede acceder a una infraestructura de computación, almacenamiento y red flexible y de alta disponibilidad, que se adapta a las necesidades y al crecimiento de la aplicación. Además, esta proporciona una variedad de herramientas y servicios que facilitan la integración, el monitoreo y la optimización del rendimiento del backend. En esta tesis se analizarán las ventajas y los desafíos de usar AWS para el hosteo del backend de la aplicación, así como las mejores prácticas y las lecciones aprendidas durante el proceso.


% Insertar seccion sofware
\clearpage

\subsection{Aplicación movil}
Esta sección presenta una visión general de la arquitectura propuesta para el sistema de Smart Parking, contextualizando brevemente el sistema multiagente a integrar y las razones detrás de la utilización de tecnologías en la nube. También se describe la estructura, detallando el papel de los componentes principales del sistema.

