\clearpage
\section{Resultados}{\label{title:resultados}}
En este proyecto, abordamos la propuesta, diseño y desarrollo de un prototipo funcional de \textit{Smart Parking}, con el objetivo de brindar a los usuarios finales la capacidad de encontrar una plaza de estacionamiento cercana a su destino deseado mediante un sistema de reservas utilizando la aplicación que hemos construido.

En el capítulo anterior (Capítulo 4), se detalló el desarrollo de los cuatro módulos fundamentales del sistema:

\begin{itemize}
    \item Nodo sensor
    \item Nodo concentrador
    \item Sistema de Reservas (servidor)
    \item Aplicación Móvil
\end{itemize}

En este capítulo, nos enfocaremos en la validación y resultados obtenidos a través de las diferentes pruebas realizadas. Estas pruebas se han dividido en tres categorías:
\begin{itemize}
    \item Pruebas de Sensores
    \item Pruebas de Comunicación y Carga del Concentrador
    \item Integración de la Aplicación con el Sistema
\end{itemize}

A continuación, describiremos en detalle cada una de estas categorías de pruebas y presentaremos los hallazgos y resultados que surgieron durante el desarrollo del proyecto.

\subsection{Pruebas de Sensores}\label{title:pruebas_sensores}

Como se detalló en la sección de Elección de Sensores (ver: \ref{title:eleccion_sensores}), al validar los tipos de sensores a utilizar, llevamos a cabo un experimento en el garaje de uno de los miembros del equipo. Esto se realizó para medir el comportamiento de los sensores, y repetimos estas pruebas en varias ocasiones para obtener las muestras que se presentarán a continuación.

Para comenzar, montamos los sensores en nuestra placa de desarrollo. Iniciamos las pruebas utilizando vehículos estacionados de frente. Es importante señalar que, para esta sección, solo se mostrará cómo interactúa el sistema cuando el automóvil se estaciona de frente. El mismo experimento se repitió también para cuando el automóvil estaciona en reversa.

\begin{images}[\label{experimento_auto}]{Diferentes momentos del vehículo durante el experimento}
    \addimage[\label{experimento_auto:inicial}]{secciones/section_5/images/auto_momento_inicial}{width=8cm}{Posición inicial del vehículo}
    \addimage[\label{experimento_auto:ocupacion}]{secciones/section_5/images/auto_momento_ocupado_perspectiva}{width=8cm}{Vehículo estacionado}
    \addimage[\label{experimento_auto:libre}]{secciones/section_5/images/auto_momento_final}{width=8cm}{Vehículo en proceso de retirada}
\end{images}

En la fase inicial del experimento, el vehículo se encontraba en la posición que se muestra en la figura \ref{experimento_auto:inicial}. En este punto, el sensor de tiempo de vuelo (ToF) tenía un valor claro que indicaba que el objeto estaba fuera de rango. Este valor se representa en el recuadro A de la figura \ref{plot_labeled:vlx}. De manera similar, el magnetómetro mantenía un nivel constante del campo magnético, tal como se observa en \ref{plot_labeled:qmc}.

\begin{images}[\label{plot_labeled}]{Diferentes momentos del vehículo durante el experimento}
    \addimage[\label{plot_labeled:vlx}]{secciones/section_5/images/vl53X0_indicaciones}{width=8cm}{Muestras del sensor VL53L0X}
    \addimage[\label{plot_labeled:qmc}]{secciones/section_5/images/qmc_edited}{width=8cm}{Muestras del sensor QMC5883L}
\end{images}

A medida que el automóvil se posicionaba sobre el sensor (\ref{experimento_auto:ocupacion}), observamos que el sensor láser (\ref{plot_labeled:vlx}) comenzó a medir la distancia desde el sensor al chasis del vehículo. Esta distancia osciló alrededor de los $180 mm$, un valor constante que depende del tipo de vehículo debido a las diferentes alturas de los chasis. También observamos una fuerte variación en el campo magnético registrado por el magnetómetro, que luego se estabilizó, como se muestra en la situación descrita por el sensor láser. Ambas situaciones mencionadas aquí se refieren al recuadro \quotes{B} de ambas imágenes.

Finalmente, cuando el vehículo comenzó a retirarse (\ref{experimento_auto:libre}), nuevamente se detectaron variaciones, que se describen en el recuadro \quotes{C}.

Cabe destacar que las lecturas del magnetómetro, como se observa en la figura \ref{plot_labeled:qmc}, eran ruidosas, lo que dificultaba la definición de un umbral de detección. Fue durante este experimento que implementamos el \textit{Detector de Anomalías}, que se puede ver en \ref{detector_anomalias}.


\subsection{Pruebas de Comunicación y Carga del Concentrador}\label{title:cx_load_concentrador}



\subsection{Búsqueda y reserva de una plaza}
Para hacer una reserva el usuario puede navegar en el mapa, o bien utilizar la barra de búsqueda para introducir su destino. Una vez que el mismo esté seleccionado, se realiza una consulta en el backend para poder verificar cuales estacionamientos cuentan aún con disponibilidad y mostrarselos al usuario. Como puede verse en la Figura ~\ref{busqueda_process}.

\begin{images}[\label{busqueda_process}]{Proceso de busqueda de estacionamientos disponibles}
    \addimage[\label{busqueda_1}]{secciones/section_5/images/busqueda_1}{width=7cm}{}
    \addimage[\label{busqueda_2}]{secciones/section_5/images/busqueda_2}{width=7cm}{}
    \addimage[\label{busqueda_3}]{secciones/section_5/images/busqueda_3}{width=7cm}{}
\end{images}

Ya el usuario habiendo seleccionado el estacionamiento que desea, completa con el tipo de reserva que quiere elegir, teniendo la posibilidad de que la reserva sea fija según un esquema de tiempo definido o libre.

\begin{images}[\label{reserva_process}]{Selección del tipo de reserva (libre o limitada)}
    \addimage[\label{reserva_limitada}]{secciones/section_5/images/reserva_limitada}{width=4cm}{}
\end{images}

Finalmente se envía esa información al backend el cual valida e informa el estado de la reserva al usuario.

\begin{images}[\label{reservacion_checkout}]{Información sobre el estado de la reserva}
    \addimage[\label{reserva_checkout_1}]{secciones/section_5/images/reserva_checkout_1}{width=4cm}{}
    \addimage[\label{reserva_checkout_2}]{secciones/section_5/images/reserva_checkout_2}{width=4cm}{}
    \addimage[\label{reserva_checkout_4}]{secciones/section_5/images/reserva_checkout_4}{width=4cm}{}
\end{images}


% SUB-SECCIÓN
\subsection{Reservaciones en curso e historial}
En cualquier aplicación orientada a la reserva de servicios, resulta fundamental contar con una sección que permita a los usuarios visualizar tanto sus reservaciones actuales como las históricas. Esta función proporciona una perspectiva completa de las reservas realizadas y en curso, ofreciendo una experiencia integral de seguimiento y gestión de las mismas.

Basándonos en el contexto anteriormente descrito, hemos desarrollado un componente destinado a la visualización de estas reservas, el cual se puede apreciar en la imagen \ref{vista_reservaciones}

\begin{images}[\label{vista_reservaciones}]{Vista reservaciones en curso e historicas}
    \addimage[\label{reservacion_en_curso}]{secciones/section_5/images/reservacion_en_curso}{width=5cm}{}
    \addimage[\label{reservacion_historica}]{secciones/section_5/images/reservacion_historica}{width=5cm}{}
\end{images}

\subsubsection{Vista detallada de la reserva}
Tanto para las reservaciones en curso como para las historicas, al ingresar en un item el usuario puede tener una vista detallada de la reserva con algunos datos de interes y un codigo QR generado que podría ser utilizado a futuro en una automatización del ingreso como se detalla en \ref{automatizacion_ingreso}

\insertimage[\label{vista_detallada_reserva}]{secciones/section_5/images/vista_detallada_reserva}{width=5cm}{Vista detallada de la reserva}

\subsubsection{Cancelación de la reserva}

Finalmente, en caso de que el usuario cambie de opinión después de haber realizado la reserva, se le brinda la opción de cancelarla directamente desde la interfaz. En un entorno de producción, podría ser contemplado dentro del modelo de negocio implementar reembolsos en base al momento en que se efectúa la cancelación. Aunque este aspecto no se aborda en detalle en este proyecto final, se menciona como una nota relevante para los lectores interesados.

\begin{images}[\label{cancelacion_reserva}]{Proceso de cancelación de una reserva}
    \addimage[\label{cancelacion_1}]{secciones/section_5/images/cancelacion_1}{width=5cm}{}
    \addimage[\label{cancelacion_2}]{secciones/section_5/images/cancelacion_2}{width=5cm}{}
    \addimage[\label{cancelacion_3}]{secciones/section_5/images/cancelacion_3}{width=5cm}{}
\end{images}
